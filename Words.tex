\documentclass{article}

\usepackage{amssymb}
\usepackage[utf8]{inputenc}
\usepackage[greek,english]{babel}
\usepackage{alphabeta}
\usepackage{titling}
\usepackage[hidelinks]{hyperref}
\usepackage{listings}
\setlength{\droptitle}{-2cm}
\usepackage[a4paper,top=3cm,bottom=2cm,left=2.75cm,right=2.75cm,marginparwidth=1.75cm]{geometry}

\author{Κωνσταντίνος Αεράκης\\ {\small Computer Science Department, University of Piraeus}\\ {\tiny https://github.com/Constantine9731/}}
\title{My very first {\LaTeX} document: The Dictionary of Shame}
\date{\textit{Anno Domini, MMXIX}}

\lstset{
basicstyle=\small\rmfamily,
columns=fullflexible,
breaklines=true,
escapeinside={(*}{*)}
}

\begin{document}
	\maketitle
%	\renewcommand{\labelitemi}{$\blacksquare$}
%	\renewcommand\labelitemii{$\square$}
	\renewcommand{\labelenumi}{\Roman{enumi}}
	\begin{itemize}

\item[$\square$] \emph{ sanctimonious }
\begin{enumerate}
\item{
\begin{lstlisting}
(* 
Επίθ. που παριστάνει το θρήσκο, ψευδευλαβής *)
\end{lstlisting}}
\item{
\begin{lstlisting}
Sanctimonious \Sanc`ti*mo"ni*ous\, a. [See {Sanctimony}.]
   1. Possessing sanctimony; holy; sacred; saintly. --Shak.
      [1913 Webster]
   2. Making a show of sanctity; affecting saintliness;
      hypocritically devout or pious. ``Like the sanctimonious
      pirate.'' --Shak.
      [1913 Webster] -- {Sanc`ti*mo"ni*ous*ly}, adv. --
      {Sanc`ti*mo"ni*ous*ness}, n.
      [1913 Webster]
\end{lstlisting}}
\end{enumerate}
\item[$\square$] \emph{ besmirch }
\begin{enumerate}
\item{
\begin{lstlisting}
(*  *)
\end{lstlisting}}
\item{
\begin{lstlisting}
Besmirch \Be*smirch"\, v. t. [imp. & p. p. {Besmirched}; p. pr.
   & vb. n. {Besmirching}.]
   To smirch or soil; to discolor; to obscure. Hence: To
   dishonor; to sully. --Shak.
   [1913 Webster]
\end{lstlisting}}
\end{enumerate}
\item[$\square$] \emph{ trudge }
\begin{enumerate}
\item{
\begin{lstlisting}
(* 
Ρημ. περπατώ κουρασμένα, περπατώ βαριά, σέρνομαι *)
\end{lstlisting}}
\item{
\begin{lstlisting}
Trudge \Trudge\, v. i. [imp. & p. p. {Trudged}; p. pr. & vb. n.
   {Trudging}.] [Perhaps of Scand. origin, and originally
   meaning, to walk on snowshoes; cf. dial. Sw. truga, trudja, a
   snowshoe, Norw. truga, Icel. [thorn]r[=u]ga.]
   To walk or march with labor; to jog along; to move wearily.
   [1913 Webster]
         And trudged to Rome upon my naked feet.  --Dryden.
   [1913 Webster]
\end{lstlisting}}
\end{enumerate}
\item[$\square$] \emph{ sardonic }
\begin{enumerate}
\item{
\begin{lstlisting}
(* 
Επίθ. σαρδόνιος, κυνικός, σαρκαστικός *)
\end{lstlisting}}
\item{
\begin{lstlisting}
Sardonic \Sar*don"ic\, a.
   Of, pertaining to, or resembling, a kind of linen made at
   Colchis.
   [1913 Webster]
Sardonic \Sar*don"ic\, a. [F. sardonique, L. sardonius, Gr. ?,
   ?, perhaps fr. ? to grin like a dog, or from a certain plant
   of Sardinia, Gr. ?, which was said to screw up the face of
   the eater.]
   Forced; unnatural; insincere; hence, derisive, mocking,
   malignant, or bitterly sarcastic; -- applied only to a laugh,
   smile, or some facial semblance of gayety.
   [1913 Webster]
         Where strained, sardonic smiles are glozing still,
         And grief is forced to laugh against her will. --Sir H.
                                                  Wotton.
   [1913 Webster]
         The scornful, ferocious, sardonic grin of a bloody
         ruffian.                                 --Burke.
   [1913 Webster]
   {Sardonic grin} or {Sardonic laugh}, an old medical term for
      a spasmodic affection of the muscles of the face, giving
      it an appearance of laughter.
      [1913 Webster]
\end{lstlisting}}
\end{enumerate}
\item[$\square$] \emph{ interstice }
\begin{enumerate}
\item{
\begin{lstlisting}
(* 
Ουσ. μικρό διάκενο *)
\end{lstlisting}}
\item{
\begin{lstlisting}
Interstice \In*ter"stice\ (?; 277), n.; pl. {Interstices}. [L.
   interstitium a pause, interval; inter between + sistere to
   set, fr. stare to stand: cf. F. interstice. See {Stand}.]
   [1913 Webster]
   1. That which intervenes between one thing and another;
      especially, a space between things closely set, or between
      the parts which compose a body; a narrow chink; a crack; a
      crevice; a hole; an interval; as, the interstices of a
      wall.
      [1913 Webster]
   2. An interval of time; specifically (R. C. Ch.), in the
      plural, the intervals which the canon law requires between
      the reception of the various degrees of orders.
      [1913 Webster]
            Nonobservance of the interstices . . . is a sin.
                                                  --Addis &
                                                  Arnold.
      [1913 Webster]
\end{lstlisting}}
\end{enumerate}
\item[$\square$] \emph{ effervesce }
\begin{enumerate}
\item{
\begin{lstlisting}
(*  *)
\end{lstlisting}}
\item{
\begin{lstlisting}
Effervesce \Ef`fer*vesce"\, v. i. [imp. & p. p. {Effervesced};
   p. pr. & vb. n. {Effervescing}.] [L. effervescere; ex +
   fervescere to begin boiling, incho., fr. fervere to boil. See
   {Fervent}.]
   1. To be in a state of natural ebullition; to bubble and
      hiss, as fermenting liquors, or any fluid, when some part
      escapes in a gaseous form.
      [1913 Webster]
   2. To exhibit, in lively natural expression, feelings that
      can not be repressed or concealed; as, to effervesce with
      joy or merriment.
\end{lstlisting}}
\end{enumerate}
\item[$\square$] \emph{ ebullition }
\begin{enumerate}
\item{
\begin{lstlisting}
(* 
Ουσ. βρασμός, αναβρασμός *)
\end{lstlisting}}
\item{
\begin{lstlisting}
Ebullition \Eb`ul*li"tion\, n. [F. ['e]bullition, L. ebullitio,
   fr. ebullire. See {Ebullient}.]
   1. A boiling or bubbling up of a liquid; the motion produced
      in a liquid by its rapid conversion into vapor.
      [1913 Webster]
   2. Effervescence occasioned by fermentation or by any other
      process which causes the liberation of a gas or an
      a["e]riform fluid, as in the mixture of an acid with a
      carbonated alkali. [Formerly written {bullition}.]
      [1913 Webster]
   3. A sudden burst or violent display; an outburst; as, an
      ebullition of anger or ill temper.
      [1913 Webster]
\end{lstlisting}}
\end{enumerate}
\item[$\square$] \emph{ cerebrospinal }
\begin{enumerate}
\item{
\begin{lstlisting}
(*  *)
\end{lstlisting}}
\item{
\begin{lstlisting}
Cerebro-spinal \Cer`e*bro-spi"nal\, a. [Cerebrum + spinal.]
   (Anat.)
   Of or pertaining to the central nervous system consisting of
   the brain and spinal cord.
   [1913 Webster]
   {Cerebro-spinal fluid} (Physiol.), a serous fluid secreted by
      the membranes covering the brain and spinal cord.
   {Cerebro-spinal meningitis}, {Cerebro-spinal fever} (Med.), a
      dangerous epidemic, and endemic, febrile disease,
      characterized by inflammation of the membranes of the
      brain and spinal cord, giving rise to severe headaches,
      tenderness of the back of the neck, paralysis of the
      ocular muscles, etc. It is sometimes marked by a cutaneous
      eruption, when it is often called spotted fever. It is not
      contagious.
      [1913 Webster]
\end{lstlisting}}
\end{enumerate}
\item[$\square$] \emph{ serendipity }
\begin{enumerate}
\item{
\begin{lstlisting}
(* 
Ουσ. ικανότητα να ανακαλύπτεις πολύτιμα πράγματα κατά τύχη, εύνοια στο να ανακαλύπτεις κάτι πολύτιμο *)
\end{lstlisting}}
\item{
\begin{lstlisting}

\end{lstlisting}}
\end{enumerate}
\item[$\square$] \emph{ impetus }
\begin{enumerate}
\item{
\begin{lstlisting}
(* 
Ουσ. ορμή, ροπή, φόρα// κινητήρια δύναμη πχ: Τhe measure gave a new impetus to trade *)
\end{lstlisting}}
\item{
\begin{lstlisting}
Impetus \Im"pe*tus\ ([i^]m"p[-e]*t[u^]s), n. [L., fr. impetere
   to rush upon, attack; pref. im- in + petere to fall upon,
   seek. See {Petition}.]
   1. A property possessed by a moving body in virtue of its
      weight and its motion; the force with which any body is
      driven or impelled; momentum.
      [1913 Webster]
   Note: Momentum is the technical term, impetus its popular
         equivalent, yet differing from it as applied commonly
         to bodies moving or moved suddenly or violently, and
         indicating the origin and intensity of the motion,
         rather than its quantity or effectiveness.
         [1913 Webster]
   2. Fig.: Impulse; incentive; stimulus; vigor; force; as, the
      President's strong recommendation provided the impetus
      needed to pass the campaign reform bill. --Buckle.
      [1913 Webster +PJC]
   3. (Gun.) The altitude through which a heavy body must fall
      to acquire a velocity equal to that with which a ball is
      discharged from a piece.
      [1913 Webster]
\end{lstlisting}}
\end{enumerate}
\item[$\square$] \emph{ concoction }
\begin{enumerate}
\item{
\begin{lstlisting}
(* 
Ουσ. παρασκευή, μαγείρεμα ανομοιογενών υλικών// παρασκεύασμα// επινόηση, σκάρφισμα *)
\end{lstlisting}}
\item{
\begin{lstlisting}
Concoction \Con*coc"tion\, n. [L. concoctio.]
   1. A change in food produced by the organs of nutrition;
      digestion. [Obs.]
      [1913 Webster]
   2. The act of concocting or preparing by combining different
      ingredients; also, the food or compound thus prepared.
      [1913 Webster]
   3. The act of digesting in the mind; planning or devising;
      rumination. --Donne.
      [1913 Webster]
   4. (Med.) Abatement of a morbid process, as a fever and
      return to a normal condition. [Obs.]
      [1913 Webster]
   5. The act of perfecting or maturing. [Obs.] --Bacon.
      [1913 Webster]
\end{lstlisting}}
\end{enumerate}
\item[$\square$] \emph{ custodian }
\begin{enumerate}
\item{
\begin{lstlisting}
(* 
Ουσ. επιστάτης, φύλακας,// φρουρός *)
\end{lstlisting}}
\item{
\begin{lstlisting}
Custodian \Cus*to"di*an\ (k?s-t?"d?-an), n. [From {Custody}.]
   One who has care or custody, as of some public building; a
   keeper or superintendent.
   [1913 Webster]
\end{lstlisting}}
\end{enumerate}
\item[$\square$] \emph{ vilify }
\begin{enumerate}
\item{
\begin{lstlisting}
(* 
Ρημ. ταπεινώνω, κακολογώ, κουρελιάζω (μτφ.) *)
\end{lstlisting}}
\item{
\begin{lstlisting}
Vilify \Vil"i*fy\, v. t. [imp. & p. p. {Vilified}; p. pr. & vb.
   n. {Vilifying}.] [L. vilis vile + -fly; cf. L. vilificare to
   esteem of little value.]
   1. To make vile; to debase; to degrade; to disgrace. [R.]
      [1913 Webster]
            When themselves they vilified
            To serve ungoverned appetite.         --Milton.
      [1913 Webster]
   2. To degrade or debase by report; to defame; to traduce; to
      calumniate. --I. Taylor.
      [1913 Webster]
            Many passions dispose us to depress and vilify the
            merit of one rising in the esteem of mankind.
                                                  --Addison.
      [1913 Webster]
   3. To treat as vile; to despise. [Obs.]
      [1913 Webster]
            I do vilify your censure.             --Beau. & Fl.
      [1913 Webster]
\end{lstlisting}}
\end{enumerate}
\item[$\square$] \emph{ thaw }
\begin{enumerate}
\item{
\begin{lstlisting}
(* 
Ουσ. λιώσιμο, τήξη, // εποχή που λιώνουν οι πάγοι ή τα χιόνια// μαλάκωμα (μτφ. για συμπεριφορά) Ρημ. επιφέρω ή υφίσταμαι τήξη// μαλακώνω (μτφ. για συμπεριφορά) *)
\end{lstlisting}}
\item{
\begin{lstlisting}
thaw \thaw\, n.
   The melting of ice, snow, or other congealed matter; the
   resolution of ice, or the like, into the state of a fluid;
   liquefaction by heat of anything congealed by frost; also, a
   warmth of weather sufficient to melt that which is congealed.
   --Dryden.
   [1913 Webster]
thaw \thaw\ (th[add]), v. i. [imp. & p. p. {Thawed} (th[add]d);
   p. pr. & vb. n. {Thawing}.] [AS. [thorn][=a]wian,
   [thorn][=a]wan; akin to D. dovijen, G. tauen, thauen (cf.
   also verdauen to digest, OHG. douwen, firdouwen), Icel.
   [thorn]eyja, Sw. t["o]a, Dan. t["o]e, and perhaps to Gr.
   th`kein to melt. [root]56.]
   1. To melt, dissolve, or become fluid; to soften; -- said of
      that which is frozen; as, the ice thaws.
      [1913 Webster]
   2. To become so warm as to melt ice and snow; -- said in
      reference to the weather, and used impersonally.
      [1913 Webster]
   3. Fig.: To grow gentle or genial. Compare {cold}[4], a. and
      {hard}[6], a.
      [1913 Webster +PJC]
thaw \thaw\, v. t.
   To cause (frozen things, as earth, snow, ice) to melt,
   soften, or dissolve.
   [1913 Webster]
\end{lstlisting}}
\end{enumerate}
\item[$\square$] \emph{ inertia }
\begin{enumerate}
\item{
\begin{lstlisting}
(* 
αδράνεια, // ατονία// νωθρότητα. *)
\end{lstlisting}}
\item{
\begin{lstlisting}
Inertia \In*er"ti*a\, n. [L., idleness, fr. iners idle. See
   {Inert}.]
   [1913 Webster]
   1. (Physics) That property of matter by which it tends when
      at rest to remain so, and when in motion to continue in
      motion, and in the same straight line or direction, unless
      acted on by some external force; -- sometimes called {vis
      inerti[ae]}. The inertia of a body is proportional to its
      mass.
      [1913 Webster +PJC]
   2. Inertness; indisposition to motion, exertion, or action;
      lack of energy; sluggishness.
      [1913 Webster]
            Men . . . have immense irresolution and inertia.
                                                  --Carlyle.
      [1913 Webster]
   3. (Med.) Lack of activity; sluggishness; -- said especially
      of the uterus, when, in labor, its contractions have
      nearly or wholly ceased.
      [1913 Webster]
   {Center of inertia}. (Mech.) See under {Center}.
      [1913 Webster]
\end{lstlisting}}
\end{enumerate}
\item[$\square$] \emph{ haughty }
\begin{enumerate}
\item{
\begin{lstlisting}that may make your noble intentions wasted.
(* 
Επιθ. υπερόπτης, αλαζόνας, επηρμένος,  ψηλομύτης *)
\end{lstlisting}}
\item{
\begin{lstlisting}
Haughty \Haugh"ty\ (h[add]"t[y^]), a. [Compar. {Haughtier}
   (h[add]"t[i^]*[~e]r); superl. {Haughtiest}.] [OE. hautein, F.
   hautain, fr. haut high, OF. also halt, fr. L. altus. See
   {Altitude}.]
   [1913 Webster]
   1. High; lofty; bold. [Obs. or Archaic]
      [1913 Webster]
            To measure the most haughty mountain's height.
                                                  --Spenser.
      [1913 Webster]
            Equal unto this haughty enterprise.   --Spenser.
      [1913 Webster]
   2. Disdainfully or contemptuously proud; arrogant;
      overbearing.
      [1913 Webster]
            A woman of a haughty and imperious nature.
                                                  --Clarendon.
      [1913 Webster]
   3. Indicating haughtiness; as, a haughty carriage.
      [1913 Webster]
            Satan, with vast and haughty strides advanced,
            Came towering.                        --Milton.
      [1913 Webster]
\end{lstlisting}}
\end{enumerate}
\item[$\square$] \emph{ phthisis }
\begin{enumerate}
\item{
\begin{lstlisting}
(*  *)
\end{lstlisting}}
\item{
\begin{lstlisting}
Phthisis \Phthi"sis\, n. [L., fr. Gr. ?, fr. ? to pass or waste
   away: cf. F. phthisie.] (Med.)
   A wasting or consumption of the tissues. The term was
   formerly applied to many wasting diseases, but is now usually
   restricted to pulmonary phthisis, or consumption. See
   {Consumption}.
   [1913 Webster]
   {Fibroid phthisis}. See under {Fibroid}.
      [1913 Webster]
\end{lstlisting}}
\end{enumerate}
\item[$\square$] \emph{ palpable }
\begin{enumerate}
\item{
\begin{lstlisting}
(* 
Επίθ. απτός, χειροπιαστός// προφανής, έκδηλος *)
\end{lstlisting}}
\item{
\begin{lstlisting}
Palpable \Pal"pa*ble\, a. [F. palpable, L. palpabilis, fr.
   palpare to feel, stroke; cf. palpus the soft palm of the
   hand.]
   1. Capable of being touched and felt; perceptible by the
      touch; as, a palpable form. --Shak.
      [1913 Webster]
            Darkness must overshadow all his bounds,
            Palpable darkness.                    --Milton.
      [1913 Webster]
   2. Easily perceptible; plain; distinct; obvious; readily
      perceived and detected; gross; as, palpable imposture;
      palpable absurdity; palpable errors. ``Three persons
      palpable.'' --P. Plowman.
      [1913 Webster]
            [Lies] gross as a mountain, open, palpable. --Shak.
      [1913 Webster]
            A hit, A very palpable hit.           --Shak.
                                                  (Hamlet)
      [1913 Webster] -- {Pal"pa*ble*ness}, n. -- {Pal"pa*bly},
      adv.
      [1913 Webster]
\end{lstlisting}}
\end{enumerate}
\item[$\square$] \emph{ delectably }
\begin{enumerate}
\item{
\begin{lstlisting}
(* 
Επίρρ. απολαυστικά, ευχάριστα *)
\end{lstlisting}}
\item{
\begin{lstlisting}
Delectable \De*lec"ta*ble\, a. [OF. delitable, OF. delitable, F.
   d['e]lectable, fr. L. delectabilis, fr. delectare to delight.
   See {Delight}.]
   1. Highly pleasing; delightful.
      [1913 Webster]
            Delectable both to behold and taste.  --Milton.
   2. extremely pleasing to the sense of taste; same as
      {luscious}, 1.
   Syn: delicious, luscious, pleasant-tasting, scrumptious,
        toothsome, yummy.
        [WordNet 1.5] -- {De*lec"ta*ble*ness}, n. --
        {De*lec"ta*bly}, adv. -- {De*lec`ta*bil"i*ty}, n.
        [1913 Webster]
\end{lstlisting}}
\end{enumerate}
\item[$\square$] \emph{ rigmarole }
\begin{enumerate}
\item{
\begin{lstlisting}
(* 
Ουσ. ανοησία , μπούρδα(ες) *)
\end{lstlisting}}
\item{
\begin{lstlisting}
Rigmarole \Rig"ma*role\, n. [For ragman roll. See {Ragman's
   roll}.]
   A succession of confused or nonsensical statements; foolish
   talk; nonsense. [Colloq.]
   [1913 Webster]
         Often one's dear friend talks something which one
         scruples to call rigmarole.              --De Quincey.
   [1913 Webster]
Rigmarole \Rig"ma*role\, a.
   Consisting of rigmarole; frivolous; nonsensical; foolish.
   [1913 Webster]
\end{lstlisting}}
\end{enumerate}
\item[$\square$] \emph{ tandem }
\begin{enumerate}
\item{
\begin{lstlisting}
(* 
Ουσ. μικρό αμαξάκι, Επίρρ. στη σειρά , διαδοχικα, ο ένας πίσω από τον άλλο Επίθ. συνεργαζόμενος στενα *)
\end{lstlisting}}
\item{
\begin{lstlisting}
Tandem \Tan"dem\, adv. & a. [L. tandem at length (of time only),
   punningly taken as meaning, lengthwise.]
   One after another; -- said especially of horses harnessed and
   driven one before another, instead of abreast.
   [1913 Webster]
Tandem \Tan"dem\, n.
   1. A team of horses harnessed one before the other. ``He
      drove tandems.'' --Thackeray.
      [1913 Webster]
   2. A tandem bicycle or other vehicle.
      [Webster 1913 Suppl.]
   {Tandem bicycle} or {Tandem tricycle}, one for two persons in
      which one rider sits before the other.
      [1913 Webster]
\end{lstlisting}}
\end{enumerate}
\item[$\square$] \emph{ foible }
\begin{enumerate}
\item{
\begin{lstlisting}
(* 
Ουσ. αδύνατο σημείο, αδυναμία *)
\end{lstlisting}}
\item{
\begin{lstlisting}
Foible \Foi"ble\, a. [OF. foible. See {Feeble}.]
   Weak; feeble. [Obs.] --Lord Herbert.
   [1913 Webster]
Foible \Foi"ble\, n.
   1. A moral weakness; a failing; a weak point; a frailty.
      [1913 Webster]
            A disposition radically noble and generous, clouded
            and overshadowed by superficial foibles. --De
                                                  Quincey.
      [1913 Webster]
   2. The half of a sword blade or foil blade nearest the point;
      -- opposed to {forte}. [Written also {faible}.]
   Syn: Fault; imperfection; failing; weakness; infirmity;
        frailty; defect. See {Fault}.
        [1913 Webster]
\end{lstlisting}}
\end{enumerate}
\item[$\square$] \emph{ infirmity }
\begin{enumerate}
\item{
\begin{lstlisting}
(* 
Ουσ. αδυναμία, αναπηρία *)
\end{lstlisting}}
\item{
\begin{lstlisting}
Infirmity \In*firm"i*ty\, n.; pl. {Infirmities}. [L. infirmitas
   : cf. F. infirmite. See {Infirm}, a.]
   1. The state of being infirm; feebleness; an imperfection or
      weakness; esp., an unsound, unhealthy, or debilitated
      state; a disease; a malady; as, infirmity of body or mind.
      [1913 Webster]
            'T is the infirmity of his age.       --Shak.
      [1913 Webster]
   2. A personal frailty or failing; foible; eccentricity; a
      weakness or defect.
      [1913 Webster]
            Will you be cured of your infirmity ? --Shak.
      [1913 Webster]
            A friend should bear his friend's infirmities.
                                                  --Shak.
      [1913 Webster]
            The house has also its infirmities.   --Evelyn.
   Syn: Debility; imbecility; weakness; feebleness; failing;
        foible; defect; disease; malady. See {Debility}.
        [1913 Webster]
\end{lstlisting}}
\end{enumerate}
\item[$\square$] \emph{ decrepitude }
\begin{enumerate}
\item{
\begin{lstlisting}
(*  *)
\end{lstlisting}}
\item{
\begin{lstlisting}
Decrepitude \De*crep"i*tude\, n. [Cf. F. d['e]cr['e]pitude.]
   The broken state produced by decay and the infirmities of
   age; infirm old age.
   [1913 Webster] ||
\end{lstlisting}}
\end{enumerate}
\item[$\square$] \emph{ debility }
\begin{enumerate}
\item{
\begin{lstlisting}
(* 
αδυναμία, εξασθένηση (ιδιαίτερα εξαιτίας ασθένειας) *)
\end{lstlisting}}
\item{
\begin{lstlisting}
Debility \De*bil"i*ty\, n. [L. debilitas, fr. debilis weak,
   prob. fr. de- + habilis able: cf. F. d['e]bilit['e]. See
   {Able}, a.]
   The state of being weak; weakness; feebleness; languor.
   [1913 Webster]
         The inconveniences of too strong a perspiration, which
         are debility, faintness, and sometimes sudden death.
                                                  --Arbuthnot.
   Syn: {Debility}, {Infirmity}, {Imbecility}.
   Usage: An infirmity belongs, for the most part, to particular
          members, and is often temporary, as of the eyes, etc.
          Debility is more general, and while it lasts impairs
          the ordinary functions of nature. Imbecility attaches
          to the whole frame, and renders it more or less
          powerless. Debility may be constitutional or may be
          the result or superinduced causes; Imbecility is
          always constitutional; infirmity is accidental, and
          results from sickness or a decay of the frame. These
          words, in their figurative uses, have the same
          distinctions; we speak of infirmity of will, debility
          of body, and an Imbecility which affects the whole
          man; but Imbecility is often used with specific
          reference to feebleness of mind.
          [1913 Webster]
\end{lstlisting}}
\end{enumerate}
\item[$\square$] \emph{ divergent }
\begin{enumerate}
\item{
\begin{lstlisting}
(* 
διιστάμενος, που αποκλίνει, αντίθετος *)
\end{lstlisting}}
\item{
\begin{lstlisting}
Divergent \Di*ver"gent\, a. [Cf. F. divergent. See {Diverge}.]
   1. Receding farther and farther from each other, as lines
      radiating from one point; deviating gradually from a given
      direction; -- opposed to {convergent}.
      [1913 Webster]
   2. (Optics) Causing divergence of rays; as, a divergent lens.
      [1913 Webster]
   3. Fig.: Disagreeing from something given; differing; as, a
      divergent statement.
      [1913 Webster]
   {Divergent series}. (Math.) See {Diverging series}, under
      {Diverging}.
      [1913 Webster]
\end{lstlisting}}
\end{enumerate}
\item[$\square$] \emph{ gregarious }
\begin{enumerate}
\item{
\begin{lstlisting}
(* 
Επιθ. αγελαίος, που ζει κατά ομάδες, κοινωνικός *)
\end{lstlisting}}
\item{
\begin{lstlisting}
Gregarious \Gre*ga"ri*ous\, a. [L. gregarius, fr. grex, gregis,
   herd; cf. Gr. ? to assemble, Skr. jar to approach. Cf.
   {Congregate}, {Egregious}.]
   Habitually living or moving in flocks or herds; tending to
   flock or herd together; not habitually solitary or living
   alone. --Burke.
   [1913 Webster]
         No birds of prey are gregarious.         --Ray.
   2. (of people) enjoying companionship; sociable; not
      solitary.
   3. (of plants) growing in clusters. -- {Gre*ga"ri*ous*ly},
      adv. -- {Gre*ga"ri*ous*ness}, n.
\end{lstlisting}}
\end{enumerate}
\item[$\square$] \emph{ solemnly }
\begin{enumerate}
\item{
\begin{lstlisting}
(* 
Επιρρ. σοβαρά, επίσημα *)
\end{lstlisting}}
\item{
\begin{lstlisting}
Solemnly \Sol"emn*ly\, adv.
   In a solemn manner; with gravity; seriously; formally.
   [1913 Webster]
         There in deaf murmurs solemnly are wise. --Dryden.
   [1913 Webster]
         I do solemnly assure the reader.         --Swift.
   [1913 Webster]
\end{lstlisting}}
\end{enumerate}
\item[$\square$] \emph{ vivacious }
\begin{enumerate}
\item{
\begin{lstlisting}
(* 
Επίθ. ζωηρός, κεφάτος, γεμάτος ζωντάνια *)
\end{lstlisting}}
\item{
\begin{lstlisting}
Vivacious \Vi*va"cious\ (?; 277), a. [L. v['i]vax, -acis, fr.
   vivere to live. See {Vivid}.]
   1. Having vigorous powers of life; tenacious of life;
      long-lived. [Obs.]
      [1913 Webster]
            Hitherto the English bishops have been vivacious
            almost to wonder. . . . But five died for the first
            twenty years of her [Queen Elizabeth's] reign.
                                                  --Fuller.
      [1913 Webster]
            The faith of Christianity is far more vivacious than
            any mere ravishment of the imagination can ever be.
                                                  --I. Taylor.
      [1913 Webster]
   2. Sprightly in temper or conduct; lively; merry; as, a
      vivacious poet. ``Vivacious nonsense.'' --V. Knox.
      [1913 Webster]
   3. (Bot.) Living through the winter, or from year to year;
      perennial. [R.]
      [1913 Webster]
   Syn: Sprightly; active; animated; sportive; gay; merry;
        jocund; light-hearted.
        [1913 Webster] -- {Vi*va"cious*ly}, adv. --
        {Vi*va"cious*ness}, n.
        [1913 Webster]
\end{lstlisting}}
\end{enumerate}
\item[$\square$] \emph{ vicariously }
\begin{enumerate}
\item{
\begin{lstlisting}
(*  *)
\end{lstlisting}}
\item{
\begin{lstlisting}
Vicariously \Vi*ca"ri*ous*ly\, adv.
   In a vicarious manner.
   [1913 Webster]
\end{lstlisting}}
\end{enumerate}
\item[$\square$] \emph{ mendacious }
\begin{enumerate}
\item{
\begin{lstlisting}
(* 
Επίθ. αναληθής, ψεύτης, άτιμος, ψευδολόγος *)
\end{lstlisting}}
\item{
\begin{lstlisting}
Mendacious \Men*da"cious\, a. [L. mendax, -acis, lying, cf.
   mentiri to lie.]
   1. Given to deception or falsehood; lying; as, a mendacious
      person.
      [1913 Webster]
   2. False; counterfeit; containing falsehood; as, a mendacious
      statement.
      [1913 Webster] -- {Men*da"cious*ly}, adv. --
      {Men*da"cious*ness}, n.
      [1913 Webster]
\end{lstlisting}}
\end{enumerate}

\item[$\square$] \emph{ tinge }
\begin{enumerate}
\item{
\begin{lstlisting}
(* 
Ουσ. απόχρωση, ελαφρά βαφή, τόνος Ρημ. δίνω τόνο (σε χρώμα ή γεύση) , δίνω χροιά *)
\end{lstlisting}}
\item{
\begin{lstlisting}
Tinge \Tinge\, v. t. [imp. & p. p. {Tinged}; p. pr. & vb. n.
   {Tingeing}.] [L. tingere, tinctum, to dye, stain, wet; akin
   to Gr. ?, and perhaps to G. tunken to dip, OHG. tunch[=o]n,
   dunch[=o]n, thunk[=o]n. Cf. {Distain}, {Dunker}, {Stain},
   {Taint} a stain, to stain, {Tincture}, {Tint}.]
   To imbue or impregnate with something different or foreign;
   as, to tinge a decoction with a bitter taste; to affect in
   some degree with the qualities of another substance, either
   by mixture, or by application to the surface; especially, to
   color slightly; to stain; as, to tinge a blue color with red;
   an infusion tinged with a yellow color by saffron.
   [1913 Webster]
         His [Sir Roger's] virtues, as well as imperfections,
         are tinged by a certain extravagance.    --Addison.
   [1913 Webster]
   Syn: To color; dye; stain.
        [1913 Webster]
Tinge \Tinge\, n.
   A degree, usually a slight degree, of some color, taste, or
   something foreign, infused into another substance or mixture,
   or added to it; tincture; color; dye; hue; shade; taste.
   [1913 Webster]
         His notions, too, respecting the government of the
         state, took a tinge from his notions respecting the
         government of the church.                --Macaulay.
   [1913 Webster]
\end{lstlisting}}
\end{enumerate}
\item[$\square$] \emph{ crevasse }
\begin{enumerate}
\item{
\begin{lstlisting}
(* 
Ουσ. βαθειά ρωγμή σε ογκόπαγο, ρωγμή *)
\end{lstlisting}}
\item{
\begin{lstlisting}
Crevasse \Cre`vasse"\ (kr?`v?s"), n. [F. See {Crevice}.]
   1. A deep crevice or fissure, as in embankment; one of the
      clefts or fissure by which the mass of a glacier is
      divided.
      [1913 Webster]
   2. A breach in the levee or embankment of a river, caused by
      the pressure of the water, as on the lower Mississippi.
      [U.S.]
      [1913 Webster]
\end{lstlisting}}
\end{enumerate}
\item[$\square$] \emph{ ravine }
\begin{enumerate}
\item{
\begin{lstlisting}
(* 
Ουσ. φαράγγι, λαγκάδα, *)
\end{lstlisting}}
\item{
\begin{lstlisting}
Raven \Rav"en\ (r[a^]v"'n), n. [OF. ravine impetuosity,
   violence, F. ravine ravine. See {Ravine}, {Rapine}.] [Written
   also {ravin}, and {ravine}.]
   1. Rapine; rapacity. --Ray.
      [1913 Webster]
   2. Prey; plunder; food obtained by violence.
      [1913 Webster]
Raven \Rav"en\, v. i.
   To prey with rapacity; to be greedy; to show rapacity.
   [Written also {ravin}, and {ravine}.]
   [1913 Webster]
         Benjamin shall raven as a wolf.          --Gen. xlix.
                                                  27.
   [1913 Webster]
\end{lstlisting}}
\end{enumerate}
\item[$\square$] \emph{ rapacity }
\begin{enumerate}
\item{
\begin{lstlisting}
(* 
Ουσ. αρπακτικότητα, πλεονεξία, απληστία *)
\end{lstlisting}}
\item{
\begin{lstlisting}
Rapacity \Ra*pac"i*ty\ (r[.a]*p[a^]s"[i^]*t[y^]), n. [L.
   rapacitas: cf. F. rapacit['e]. See {Rapacious}.]
   1. The quality of being rapacious; rapaciousness;
      ravenousness; as, the rapacity of pirates; the rapacity of
      wolves.
      [1913 Webster]
   2. The act or practice of extorting or exacting by oppressive
      injustice; exorbitant greediness of gain. ``The rapacity
      of some ages.'' --Sprat.
      [1913 Webster]
\end{lstlisting}}
\end{enumerate}

\item[$\square$] \emph{ scraggy }
\begin{enumerate}
\item{
\begin{lstlisting}
(* 
Επίθ. αχαμνός, ισχνός, καχεκτικός, λιπόσαρκος *)
\end{lstlisting}}
\item{
\begin{lstlisting}
Scraggy \Scrag"gy\, a. [Compar. {Scragger}; superl.
   {Scraggiest}.]
   1. Rough with irregular points; scragged. ``A scraggy rock.''
      --J. Philips.
      [1913 Webster]
   2. Lean and rough; scragged. ``His sinewy, scraggy neck.''
      --Sir W. Scott.
      [1913 Webster]
\end{lstlisting}}
\end{enumerate}
\item[$\square$] \emph{ exude }
\begin{enumerate}
\item{
\begin{lstlisting}
(* 
Ρημ.. εξιδρώνω (ιατρ.), εκκρίνω *)
\end{lstlisting}}
\item{
\begin{lstlisting}
Exude \Ex*ude"\, v. t. [imp. & p. p. {Exuded}; p. pr. & vb. n.
   {exuding}.] [L. exudare, exsudare, exudatum, exsudatum, to
   sweat out; ex out + sudare to sweat: cf. F. exuder, exsuder.
   See {Sweat}.]
   To discharge through pores or incisions, as moisture or other
   liquid matter; to give out.
   [1913 Webster]
         Our forests exude turpentine in . . . abundance. --Dr.
                                                  T. Dwight.
   [1913 Webster]
Exude \Ex*ude"\, v. i.
   To flow from a body through the pores, or by a natural
   discharge, as juice.
   [1913 Webster]
\end{lstlisting}}
\end{enumerate}
\item[$\square$] \emph{ neuter }
\begin{enumerate}
\item{
\begin{lstlisting}
(* 
Ουσ. ουδέτερο (γραμμ.), αμετάβατο (γραμμ.)  Επιθ. άφυλος, ουδέτερος, ευνουχισμένος (για ζώα) Ρημ. ευνουχίζω, καθιστώ σεξουαλικά ανενεργό *)
\end{lstlisting}}
\item{
\begin{lstlisting}
neuter \neu"ter\, v. t.
   To render incapable of sexual reproduction; to remove or
   alter the sexual organs so as to make infertile; to alter; to
   fix; to desex; -- in male animals, to {castrate}; in female
   animals, to {spay}.
   [PJC]
\end{lstlisting}}
\end{enumerate}
\item[$\square$] \emph{ deteriorate }
\begin{enumerate}
\item{
\begin{lstlisting}
(* 
Ρημ. επιδεινώνομαι, χειροτερεύω//  εκφυλίζομαι//  χαλώ την ποιότητα *)
\end{lstlisting}}
\item{
\begin{lstlisting}
deteriorate \de*te"ri*o*rate\ (d[-e]*t[=e]"r[i^]*[-o]*r[=a]t),
   v. i.
   To grow worse; to be impaired in quality; to degenerate.
   [1913 Webster]
         Under such conditions, the mind rapidly deteriorates.
                                                  --Goldsmith.
   [1913 Webster]
\end{lstlisting}}
\end{enumerate}
\item[$\square$] \emph{ detergent }
\begin{enumerate}
\item{
\begin{lstlisting}
(* 
Ουσ. καθαριστικό , απορρυπαντικό//  απολυμαντικό, καθαρκτικό *)
\end{lstlisting}}
\item{
\begin{lstlisting}
Detergent \De*ter"gent\, a. [L. detergens, -entis, p. pr. of
   detergere: cf. F. d['e]tergent.]
   Cleansing; purging. -- n. A substance which cleanses the
   skin, as water or soap; a medicine to cleanse wounds, ulcers,
   etc.
   [1913 Webster]
\end{lstlisting}}
\end{enumerate}
\item[$\square$] \emph{ uncanny }
\begin{enumerate}
\item{
\begin{lstlisting}
(* 
Επίθ. αλλόκοτος, παράξενος, υπερβολικά μυστηριώδης *)
\end{lstlisting}}
\item{
\begin{lstlisting}
Uncanny \Un*can"ny\, a.
   Not canny; unsafe; strange; weird; ghostly. --Sir W. Scott.
   -- {Un*can"ni*ness}, n. --G. Eliot.
   [1913 Webster]
\end{lstlisting}}
\end{enumerate}

\item[$\square$] \emph{ contingent }
\begin{enumerate}
\item{
\begin{lstlisting}
(* 
Ουσ. στρατιωτικό ή αστυνομικό άγημα - απόσπασμα// αντιπροσωπεία , ομάδα εκπροσώπησης// τυχαίο συμβάν / Επίθ. τυχαίος// εξαρτώμενος (από κάτι άλλο ως προς το να συμβεί) *)
\end{lstlisting}}
\item{
\begin{lstlisting}
Contingent \Con*tin"gent\, n.
   1. An event which may or may not happen; that which is
      unforeseen, undetermined, or dependent on something
      future; a contingency.
      [1913 Webster]
            His understanding could almost pierce into future
            contingents.                          --South.
      [1913 Webster]
   2. That which falls to one in a division or apportionment
      among a number; a suitable share; proportion; esp., a
      quota of troops.
      [1913 Webster]
            From the Alps to the border of Flanders, contingents
            were required . . . 200,000 men were in arms.
                                                  --Milman.
      [1913 Webster]
Contingent \Con*tin"gent\, a. [L. contingens, -entis, p. pr. of
   contingere to touch on all sides, to happen; con- + tangere
   to touch: cf. F. contingent. See {Tangent}, {Tact}.]
   1. Possible, or liable, but not certain, to occur;
      incidental; casual.
      [1913 Webster]
            Weighing so much actual crime against so much
            contingent advantage.                 --Burke.
      [1913 Webster]
   2. Dependent on that which is undetermined or unknown; as,
      the success of his undertaking is contingent upon events
      which he can not control. ``Uncertain and contingent
      causes.'' --Tillotson.
      [1913 Webster]
   3. (Law) Dependent for effect on something that may or may
      not occur; as, a contingent estate.
      [1913 Webster]
            If a contingent legacy be left to any one when he
            attains, or if he attains, the age of twenty-one.
                                                  --Blackstone.
      [1913 Webster]
\end{lstlisting}}
\end{enumerate}
\item[$\square$] \emph{ presume }
\begin{enumerate}
\item{
\begin{lstlisting}
(* 
υποθέτω , προϋποθέτω, θεωρώ δεδομένο// εμπιστεύομαι , δέχομαι ως αλήθεια// τολμώ, παίρνω το θάρρος *)
\end{lstlisting}}
\item{
\begin{lstlisting}
Presume \Pre*sume"\, v. t. [imp. & p. p. {Presumed}; p. pr. &
   vb. n. {Presuming}.] [F. pr['e]sumer, L. praesumere,
   praesumptum; prae before + sumere to take. See {Assume},
   {Redeem}.]
   1. To assume or take beforehand; esp., to do or undertake
      without leave or authority previously obtained.
      [1913 Webster]
            Dare he presume to scorn us in this manner? --Shak.
      [1913 Webster]
            Bold deed thou hast presumed, adventurous Eve.
                                                  --Milton.
      [1913 Webster]
   2. To take or suppose to be true, or entitled to belief,
      without examination or proof, or on the strength of
      probability; to take for granted; to infer; to suppose.
      [1913 Webster]
            Every man is to be presumed innocent till he is
            proved to be guilty.                  --Blackstone.
      [1913 Webster]
            What rests but that the mortal sentence pass, . . .
            Which he presumes already vain and void,
            Because not yet inflicted?            --Milton.
      [1913 Webster]
Presume \Pre*sume"\, v. i.
   1. To suppose or assume something to be, or to be true, on
      grounds deemed valid, though not amounting to proof; to
      believe by anticipation; to infer; as, we may presume too
      far.
      [1913 Webster]
   2. To venture, go, or act, by an assumption of leave or
      authority not granted; to go beyond what is warranted by
      the circumstances of the case; to venture beyond license;
      to take liberties; -- often with on or upon before the
      ground of confidence.
      [1913 Webster]
            Do not presume too much upon my love. --Shak.
      [1913 Webster]
            This man presumes upon his parts.     --Locke.
      [1913 Webster]
\end{lstlisting}}
\end{enumerate}
\item[$\square$] \emph{ nigh }
\begin{enumerate}
\item{
\begin{lstlisting}
(*  *)
\end{lstlisting}}
\item{
\begin{lstlisting}
Nigh \Nigh\ (n[imac]), a. [Compar. {Nigher} (n[imac]"[~e]r);
   superl. {Nighest}, or {Next} (n[e^]kst).] [OE. nigh, neigh,
   neih, AS. ne['a]h, n[=e]h; akin to D. na, adv., OS. n[=a]h,
   a., OHG. n[=a]h, G. nah, a., nach to, after, Icel. n[=a] (in
   comp.) nigh, Goth. n[=e]hw, n[=e]hwa, adv., nigh. Cf. {Near},
   {Neighbor}, {Next}.]
   1. Not distant or remote in place or time; near.
      [1913 Webster]
            The loud tumult shows the battle nigh. --Prior.
      [1913 Webster]
   2. Not remote in degree, kindred, circumstances, etc.;
      closely allied; intimate. ``Nigh kinsmen.'' --Knolles.
      [1913 Webster]
            Ye . . . are made nigh by the blood of Christ.
                                                  --Eph. ii. 13.
      [1913 Webster]
   Syn: Near; close; adjacent; contiguous; present; neighboring.
        [1913 Webster]
Nigh \Nigh\, adv. [AS. ne['a]h, n[=e]h. See {Nigh}, a.]
   1. In a situation near in place or time, or in the course of
      events; near.
      [1913 Webster]
            He was sick, nigh unto death.         --Phil. ii.
                                                  27.
      [1913 Webster]
            He drew not nigh unheard; the angel bright,
            Ere he drew nigh, his radiant visage turned.
                                                  --Milton.
      [1913 Webster]
   2. Almost; nearly; as, he was nigh dead.
      [1913 Webster]
Nigh \Nigh\, v. t. & i.
   To draw nigh (to); to approach; to come near. [Obs.] --Wyclif
   (Matt. iii. 2).
   [1913 Webster]
Nigh \Nigh\, prep.
   Near to; not remote or distant from. ``was not this nigh
   shore?'' --Shak.
   [1913 Webster]
\end{lstlisting}}
\end{enumerate}
\item[$\square$] \emph{ respite }
\begin{enumerate}
\item{
\begin{lstlisting}
(* 
Ουσ. ανάπαυλα, μικρή διακοπή (από κάτι μη ευχάριστο)// αναβολή, αναστολή (μη ευχάριστου) / Ρημ. αναστέλλω, αναβάλλω *)
\end{lstlisting}}
\item{
\begin{lstlisting}
Respite \Res"pite\ (r?s"p?t), n. [OF. respit, F. r['e]pit, from
   L. respectus respect, regard, delay, in LL., the deferring of
   a day. See {Respect}.]
   1. A putting off of that which was appointed; a postponement
      or delay.
      [1913 Webster]
            I crave but four day's respite.       --Shak.
      [1913 Webster]
   2. Temporary intermission of labor, or of any process or
      operation; interval of rest; pause; delay. ``Without more
      respite.'' --Chaucer.
      [1913 Webster]
            Some pause and respite only I require. --Denham.
      [1913 Webster]
   3. (Law)
      (a) Temporary suspension of the execution of a capital
          offender; reprieve.
      (b) The delay of appearance at court granted to a jury
          beyond the proper term.
          [1913 Webster]
   Syn: Pause; interval; stop; cessation; delay; postponement;
        stay; reprieve.
        [1913 Webster]
Respite \Res"pite\, v. t. [imp. & p. p. {Respited}; p. pr. & vb.
   n. {Respiting}.] [OF. respiter, LL. respectare. See
   {Respite}, n.]
   To give or grant a respite to. Specifically:
   (a) To delay or postpone; to put off.
   (b) To keep back from execution; to reprieve.
       [1913 Webster]
             Forty days longer we do respite you. --Shak.
       [1913 Webster]
   (c) To relieve by a pause or interval of rest. ``To respite
       his day labor with repast.'' --Milton.
       [1913 Webster]
\end{lstlisting}}
\end{enumerate}
\item[$\square$] \emph{ putative }
\begin{enumerate}
\item{
\begin{lstlisting}
(* 
Επίθ. θεωρούμενος, υποτιθέμενος *)
\end{lstlisting}}
\item{
\begin{lstlisting}
Putative \Pu"ta*tive\, a. [L. putativus, fr. putare, putatum, to
   reckon, suppose, adjust, prune, cleanse. See {Pure}, and cf.
   {Amputate}, {Compute}, {Dispute}, {Impute}.]
   Commonly thought or deemed; supposed; reputed; as, the
   putative father of a child. ``His other putative (I dare not
   say feigned) friends.'' --E. Hall.
   [1913 Webster]
         Thus things indifferent, being esteemed useful or
         pious, became customary, and then came for reverence
         into a putative and usurped authority.   --Jer. Taylor.
   [1913 Webster]
\end{lstlisting}}
\end{enumerate}
\item[$\square$] \emph{ froward }
\begin{enumerate}
\item{
\begin{lstlisting}
(* 
Επίθ. αντιρρησίας, πνεύμα αντιλογίας, δύσκολος *)
\end{lstlisting}}
\item{
\begin{lstlisting}
Froward \Fro"ward\, a. [Fro + -ward. See {Fro}, and cf.
   {Fromward}.]
   Not willing to yield or compIy with what is required or is
   reasonable; perverse; disobedient; peevish; as, a froward
   child.
   [1913 Webster]
         A froward man soweth strife.             --Prov. xvi.
                                                  28.
   [1913 Webster]
         A froward retention of custom is as turbulent a thing
         as innovation.                           --Bacon.
   Syn: Untoward; wayward; unyielding; ungovernable: refractory;
        obstinate; petulant; cross; peevish. See {Perverse}. --
        {Fro"ward*ly}, adv. -- {Fro"ward*ness}, n.
        [1913 Webster]
\end{lstlisting}}
\end{enumerate}
\item[$\square$] \emph{ flustered }
\begin{enumerate}
\item{
\begin{lstlisting}
(* 
Ουσ. πελάγωμα (λόγω βιασύνης ή πολλών εργασιών) σάστισμα / Ρημ. κάνω κάποιον να πελαγώσει// προκαλώ σάστισμα *)
\end{lstlisting}}
\item{
\begin{lstlisting}
Fluster \Flus"ter\, v. t. [imp. & p. p. {Flustered}; p. pr. &
   vb. n. {Flustering}.] [Cf. Icel. flaustra to be flustered,
   flaustr a fluster.]
   To make hot and rosy, as with drinking; to heat; hence, to
   throw into agitation and confusion; to confuse; to muddle.
   [1913 Webster]
         His habit or flustering himself daily with claret.
                                                  --Macaulay.
   [1913 Webster]
\end{lstlisting}}
\end{enumerate}
\item[$\square$] \emph{ proponent }
\begin{enumerate}
\item{
\begin{lstlisting}
(* 
Ουσ. κάποιος που προτείνει μια ιδέα, ο εμπνευστής (θεωρίας ή ίδέας) *)
\end{lstlisting}}
\item{
\begin{lstlisting}
Proponent \Pro*po"nent\, a. [L. proponens, p. pr.]
   Making proposals; proposing.
   [1913 Webster]
Proponent \Pro*po"nent\, n.
   [1913 Webster]
   1. One who makes a proposal, or lays down a proposition.
      --Dryden.
      [1913 Webster]
   2. (Law) The propounder of a thing.
      [1913 Webster]
\end{lstlisting}}
\end{enumerate}
\item[$\square$] \emph{ unadulterated }
\begin{enumerate}
\item{
\begin{lstlisting}
(* 
Επίθ. ανόθευτος, πλήρης// που δεν έχει άλλες ουσίες ή προσμίξεις *)
\end{lstlisting}}
\item{
\begin{lstlisting}
Unadulterate \Un`a*dul"ter*ate\, Unadulterated
\Un`a*dul"ter*a`ted\, a.
   Not adulterated; pure. ``Unadulterate air.'' --Cowper. --
   {Un`a*dul"ter*ate*ly}, adv.
   [1913 Webster]
\end{lstlisting}}
\end{enumerate}
\item[$\square$] \emph{ depravity }
\begin{enumerate}
\item{
\begin{lstlisting}
(* 
Ουσ. διαφθορά, φαυλότητα, ροπή προς αμαρτία *)
\end{lstlisting}}
\item{
\begin{lstlisting}
Depravity \De*prav"i*ty\, n. [From {Deprave}: cf. L. pravitas
   crookedness, perverseness.]
   The state of being depraved or corrupted; a vitiated state of
   moral character; general badness of character; wickedness of
   mind or heart; absence of religious feeling and principle.
   [1913 Webster]
   {Total depravity}. See {Original sin}, and {Calvinism}.
   Syn: Corruption; vitiation; wickedness; vice; contamination;
        degeneracy.
   Usage: {Depravity}, {Depravation}, {Corruption}. Depravilty
          is a vitiated state of mind or feeling; as, the
          depravity of the human heart; depravity of public
          morals. Depravation points to the act or process of
          making depraved, and hence to the end thus reached;
          as, a gradual depravation of principle; a depravation
          of manners, of the heart, etc. Corruption is the only
          one of these words which applies to physical
          substances, and in reference to these denotes the
          process by which their component parts are dissolved.
          Hence, when figuratively used, it denotes an utter
          vitiation of principle or feeling. Depravity applies
          only to the mind and heart: we can speak of a depraved
          taste, or a corrupt taste; in the first we introduce
          the notion that there has been the influence of bad
          training to pervert; in the second, that there is a
          want of true principle to pervert; in the second, that
          there is a want of true principles to decide. The
          other two words have a wider use: we can speak of the
          depravation or the corruption of taste and public
          sentiment. Depravity is more or less open; corruption
          is more or less disguised in its operations. What is
          depraved requires to be reformed; what is corrupt
          requires to be purified.
          [1913 Webster]
\end{lstlisting}}
\end{enumerate}
\item[$\square$] \emph{ miscegenation }
\begin{enumerate}
\item{
\begin{lstlisting}
(* 
Ουσ. επιμιξία// μικτός γάμος, *)
\end{lstlisting}}
\item{
\begin{lstlisting}
Miscegenation \Mis`ce*ge*na"tion\, n. [L. miscere to mix + the
   root of genus race.]
   A mixing of races; amalgamation, as by intermarriage of black
   and white.
   Note: Until the late twentieth century, misceganation was a
         crime in some states of the Southern United States.
         [1913 Webster +PJC]
\end{lstlisting}}
\end{enumerate}
\item[$\square$] \emph{ cabletow }
\begin{enumerate}
\item{
\begin{lstlisting}
(*  *)
\end{lstlisting}}
\item{
\begin{lstlisting}
Cablet \Ca"blet\, n. [Dim. of cable; cf. F. c[^a]blot.]
   A little cable less than ten inches in circumference.
   [1913 Webster]
\end{lstlisting}}
\end{enumerate}
\item[$\square$] \emph{ festering }
\begin{enumerate}
\item{
\begin{lstlisting}
(* 
Ρημ. εμπυούμαι, κακοφορμίζω// ανάβω (για θυμό) Ουσ. πληγή με πύον *)
\end{lstlisting}}
\item{
\begin{lstlisting}
Fester \Fes"ter\, v. i. [imp. & p. p. {Festered}; p. pr. & vb.
   n. {Festering}.] [OE. festern, fr. fester, n.; or fr. OF.
   festrir, fr. festre, n. See {Fester}, n.]
   1. To generate pus; to become imflamed and suppurate; as, a
      sore or a wound festers.
      [1913 Webster]
            Wounds immedicable
            Rankle, and fester, and gangrene.     --Milton.
      [1913 Webster]
            Unkindness may give a wound that shall bleed and
            smart, but it is treachery that makes it fester.
                                                  --South.
      [1913 Webster]
            Hatred . . . festered in the hearts of the children
            of the soil.                          --Macaulay.
      [1913 Webster]
   2. To be inflamed; to grow virulent, or malignant; to grow in
      intensity; to rankle.
      [1913 Webster]
\end{lstlisting}}
\end{enumerate}
\item[$\square$] \emph{ condone }
\begin{enumerate}
\item{
\begin{lstlisting}
(* 
Ρημ. παραβλέπω, συγχωρώ, χαρίζομαι (συν. με το ζόρι - χωρίς κατά βάθος να θέλω) // ανέχομαι (πχ. απιστία συζύγου) *)
\end{lstlisting}}
\item{
\begin{lstlisting}
Condone \Con*done"\, v. t. [imp. & p. p. {Condoned}; p. pr. &
   vb. n. {Condoning}.] [L. condonare, -donatum, to give up,
   remit, forgive; con- + donare to give. See {Donate}.]
   1. To pardon; to forgive.
      [1913 Webster]
            A fraud which he had either concocted or condoned.
                                                  --W. Black.
      [1913 Webster]
            It would have been magnanimous in the men then in
            power to have overlooked all these things, and,
            condoning the politics, to have rewarded the poetry
            of Burns.                             --J. C.
                                                  Shairp.
      [1913 Webster]
   2. (Law) To pardon; to overlook the offense of; esp., to
      forgive for a violation of the marriage law; -- said of
      either the husband or the wife.
      [1913 Webster]
\end{lstlisting}}
\end{enumerate}
\item[$\square$] \emph{ stunted }
\begin{enumerate}
\item{
\begin{lstlisting}
(* 
Ουσ. εντυπωσιακό κατόρθωμα,// κάτι που γίνεται για να τραβήξει την προσοχή, διαφημιστικό τέχνασμα / Ρημ. εμποδίζω την ανάπτυξη, κατσιάζω, κολοβώνω *)
\end{lstlisting}}
\item{
\begin{lstlisting}
Stunted \Stunt"ed\, a.
   Dwarfed. -- {Stunt"ed*ness}, n.
   [1913 Webster]
Stunt \Stunt\, v. t. [imp. & p. p. {Stunted}; p. pr. & vb. n.
   {Stunting}.] [See {Stint}.]
   To hinder from growing to the natural size; to prevent the
   growth of; to stint, to dwarf; as, to stunt a child; to stunt
   a plant.
   [1913 Webster]
         When, by a cold penury, I blast the abilities of a
         nation, and stunt the growth of its active energies,
         the ill or may do is beyond all calculation. --Burke.
   [1913 Webster]
\end{lstlisting}}
\end{enumerate}
\item[$\square$] \emph{ nonchalantly }
\begin{enumerate}
\item{
\begin{lstlisting}
(* 
αδιάφορα, ήρεμα, ατάραχα, ψύχραιμα *)
\end{lstlisting}}
\item{
\begin{lstlisting}
Nonchalantly \Non"cha*lant`ly\, adv.
   In a nonchalant, indifferent, or careless manner; coolly.
   [1913 Webster]
\end{lstlisting}}
\end{enumerate}
\item[$\square$] \emph{ grandeur }
\begin{enumerate}
\item{
\begin{lstlisting}
(* 
Ουσ. μεγαλείο, αρχοντιά, μεγαλοπρέπεια, λαμπρότητα, ανωτερότητα *)
\end{lstlisting}}
\item{
\begin{lstlisting}
Grandeur \Gran"deur\, n. [F., fr. grand. See {Grand}.]
   The state or quality of being grand; vastness; greatness;
   splendor; magnificence; stateliness; sublimity; dignity;
   elevation of thought or expression; nobility of action.
   [1913 Webster]
         Nor doth this grandeur and majestic show
         Of luxury . . . allure mine eye.         --Milton.
   Syn: Sublimity; majesty; stateliness; augustness; loftiness.
        See {Sublimity}.
        [1913 Webster]
\end{lstlisting}}
\end{enumerate}
\item[$\square$] \emph{ aberration }
\begin{enumerate}
\item{
\begin{lstlisting}
(* 
Ουσ. παρέκλιση, παρεκτροπή//  παραλογισμός, διαταραχή του μυαλού. *)
\end{lstlisting}}
\item{
\begin{lstlisting}
Aberration \Ab`er*ra"tion\, n. [L. aberratio: cf. F. aberration.
   See {Aberrate}.]
   1. The act of wandering; deviation, especially from truth or
      moral rectitude, from the natural state, or from a type.
      ``The aberration of youth.'' --Hall. ``Aberrations from
      theory.'' --Burke.
      [1913 Webster]
   2. A partial alienation of reason. ``Occasional aberrations
      of intellect.'' --Lingard.
      [1913 Webster]
            Whims, which at first are the aberrations of a
            single brain, pass with heat into epidemic form.
                                                  --I. Taylor.
      [1913 Webster]
   3. (Astron.) A small periodical change of position in the
      stars and other heavenly bodies, due to the combined
      effect of the motion of light and the motion of the
      observer; called {annual aberration}, when the observer's
      motion is that of the earth in its orbit, and daily or
      {diurnal aberration}, when of the earth on its axis;
      amounting when greatest, in the former case, to 20.4'',
      and in the latter, to 0.3''. {Planetary aberration} is
      that due to the motion of light and the motion of the
      planet relative to the earth.
      [1913 Webster]
   4. (Opt.) The convergence to different foci, by a lens or
      mirror, of rays of light emanating from one and the same
      point, or the deviation of such rays from a single focus;
      called {spherical aberration}, when due to the spherical
      form of the lens or mirror, such form giving different
      foci for central and marginal rays; and {chromatic
      aberration}, when due to different refrangibilities of the
      colored rays of the spectrum, those of each color having a
      distinct focus.
      [1913 Webster]
   5. (Physiol.) The passage of blood or other fluid into parts
      not appropriate for it.
      [1913 Webster]
   6. (Law) The producing of an unintended effect by the
      glancing of an instrument, as when a shot intended for A
      glances and strikes B.
      [1913 Webster]
   Syn: Insanity; lunacy; madness; derangement; alienation;
        mania; dementia; hallucination; illusion; delusion. See
        {Insanity}.
        [1913 Webster]
\end{lstlisting}}
\end{enumerate}
\item[$\square$] \emph{ digression }
\begin{enumerate}
\item{
\begin{lstlisting}
(* 
Ουσ. παρέκλιση, παρέκβαση, παρεκτροπή (σε γράψιμο ή ομιλία) *)
\end{lstlisting}}
\item{
\begin{lstlisting}
Digression \Di*gres"sion\, n. [L. digressio: cf. F. digression.]
   1. The act of digressing or deviating, esp. from the main
      subject of a discourse; hence, a part of a discourse
      deviating from its main design or subject.
      [1913 Webster]
            The digressions I can not excuse otherwise, than by
            the confidence that no man will read them. --Sir W.
                                                  Temple.
      [1913 Webster]
   2. A turning aside from the right path; transgression;
      offense. [R.]
      [1913 Webster]
            Then my digression is so vile, so base,
            That it will live engraven in my face. --Shak.
      [1913 Webster]
   3. (Anat.) The elongation, or angular distance from the sun;
      -- said chiefly of the inferior planets. [R.]
      [1913 Webster]
\end{lstlisting}}
\end{enumerate}
\item[$\square$] \emph{ propensity }
\begin{enumerate}
\item{
\begin{lstlisting}
(* 
ροπή , τάση, κλίση, διάθεση *)
\end{lstlisting}}
\item{
\begin{lstlisting}
Propensity \Pro*pen"si*ty\, n.; pl. {Propensities}.
   The quality or state of being propense; natural inclination;
   disposition to do good or evil; bias; bent; tendency. ``A
   propensity to utter blasphemy.'' --Macaulay.
   [1913 Webster]
   Syn: Disposition; bias; inclination; proclivity; proneness;
        bent; tendency.
        [1913 Webster]
\end{lstlisting}}
\end{enumerate}
\item[$\square$] \emph{ abhorrent }
\begin{enumerate}
\item{
\begin{lstlisting}
(* 
Επιθ. αποκρουστικός, μισητός, απεχθής. *)
\end{lstlisting}}
\item{
\begin{lstlisting}
Abhorrent \Ab*hor"rent\, a. [L. abhorens, -rentis, p. pr. of
   abhorrere.]
   1. Abhorring; detesting; having or showing abhorrence;
      loathing; hence, strongly opposed to; as, abhorrent
      thoughts.
      [1913 Webster]
            The persons most abhorrent from blood and treason.
                                                  --Burke.
      [1913 Webster]
            The arts of pleasure in despotic courts
            I spurn abhorrent.                    --Clover.
      [1913 Webster]
   2. Contrary or repugnant; discordant; inconsistent; --
      followed by to. ``Injudicious profanation, so abhorrent to
      our stricter principles.'' --Gibbon.
      [1913 Webster]
   3. Detestable. ``Pride, abhorrent as it is.'' --I. Taylor.
      [1913 Webster]
\end{lstlisting}}
\end{enumerate}
\item[$\square$] \emph{ vernacular }
\begin{enumerate}
\item{
\begin{lstlisting}
(* 
Ουσ. δημώδες τοπικό ιδίωμα, τοπική διάλεκτος, τοπική λαλιά *)
\end{lstlisting}}
\item{
\begin{lstlisting}
Vernacular \Ver*nac"u*lar\, a. [L. vernaculus born in one's
   house, native, fr. verna a slave born in his master's house,
   a native, probably akin to Skr. vas to dwell, E. was.]
   Belonging to the country of one's birth; one's own by birth
   or nature; native; indigenous; -- now used chiefly of
   language; as, English is our vernacular language. ``A
   vernacular disease.'' --Harvey.
   [1913 Webster]
         His skill the vernacular dialect of the Celtic tongue.
                                                  --Fuller.
   [1913 Webster]
         Which in our vernacular idiom may be thus interpreted.
                                                  --Pope.
   [1913 Webster]
Vernacular \Ver*nac"u*lar\, n.
   The vernacular language; one's mother tongue; often, the
   common forms of expression in a particular locality.
   [1913 Webster]
\end{lstlisting}}
\end{enumerate}
\item[$\square$] \emph{ quixotic }
\begin{enumerate}
\item{
\begin{lstlisting}
(* 
Επίθ. δονκιχωτικός, που έχει ιδέες που είναι καλές αλλά δύσκολα εφαρμόζονται *)
\end{lstlisting}}
\item{
\begin{lstlisting}
Quixotic \Quix*ot"ic\ (kw[i^]ks*[o^]t"[i^]k), a.
   1. Like Don Quixote; romantic to extravagance; prone to
      pursue unrealizable goals; absurdly chivalric; apt to be
      deluded. See also {quixotism}. ``Feats of quixotic
      gallantry.'' --Prescott.
      [1913 Webster]
   2. Like the deeds of Don Quixote; ridiculously impractical;
      unachievable; extravagantly romantic; doomed to failure;
      as, a quixotic quest.
      [PJC]
            The word ``quixotic'' . . . has entered the common
            language, with the meaning ``hopelessly naive and
            idealistic,'' ``ridiculously impractical,'' ``doomed
            to fail.'' That this epithet can be used now in an
            exclusively pejorative sense not only shows that we
            have ceased to read Cervantes and to understand his
            character, but more fundamentally it reveals that
            our culture has drifted away from its spiritual
            roots.                                --Simon Leys
                                                  (N. Y. Review
                                                  of Books, June
                                                  11, 1998, p.
                                                  35)
      [PJC]
\end{lstlisting}}
\end{enumerate}
\item[$\square$] \emph{ indomitable }
\begin{enumerate}
\item{
\begin{lstlisting}
(* 
Επίθ. αδάμαστος, αλύγιστος *)
\end{lstlisting}}
\item{
\begin{lstlisting}
Indomitable \In*dom"i*ta*ble\, a. [L. indomitabilis; pref. in-
   not + domitare, intens. fr. domare to tame. See {Tame}.]
   Not to be subdued; untamable; invincible; as, an indomitable
   will, courage, animal.
   [1913 Webster]
\end{lstlisting}}
\end{enumerate}
\item[$\square$] \emph{ immaculate }
\begin{enumerate}
\item{
\begin{lstlisting}
(* 
Επιθ. άσπιλος, ακηλίδωτος, άψογος, *)
\end{lstlisting}}
\item{
\begin{lstlisting}
Immaculate \Im*mac"u*late\, a. [L. immaculatus; pref. im- not +
   maculatus, p. p. of maculare to spot, stane, fr. macula spot.
   See {Mail} armor.]
   Without stain or blemish; spotless; undefiled; clear; pure.
   [1913 Webster]
         Were but my soul as pure
         From other guilt as that, Heaven did not hold
         One more immaculate.                     --Denham.
   [1913 Webster]
         Thou sheer, immaculate and silver fountain. --Shak.
   [1913 Webster]
   {Immaculate conception} (R. C. Ch.), the doctrine that the
      Virgin Mary was conceived without original sin. --
      {Im*mac"u*late*ly}, adv. -- {Im*mac"u*late*ness}, n.
      [1913 Webster]
\end{lstlisting}}
\end{enumerate}
\item[$\square$] \emph{ sediment }
\begin{enumerate}
\item{
\begin{lstlisting}
(* 
Ουσ. ίζημα, κατακάθι *)
\end{lstlisting}}
\item{
\begin{lstlisting}
Sediment \Sed"i*ment\, n. [F. s['e]diment, L. sedimentum a
   settling, fr. sedere to sit, to settle. See {Sit}.]
   1. The matter which subsides to the bottom, from water or any
      other liquid; settlings; lees; dregs.
      [1913 Webster]
   2. (Geol.) The material of which sedimentary rocks are
      formed.
      [1913 Webster]
\end{lstlisting}}
\end{enumerate}

\item[$\square$] \emph{ facet }
\begin{enumerate}
\item{
\begin{lstlisting}
(* 
Ουσ. εδρα επεξεργασμένου πολύτιμου λίθου.// όψη //  πλευρά διαμαντιού. / Ρημ.  κόβω πολυεδρικά. *)
\end{lstlisting}}
\item{
\begin{lstlisting}
Facet \Fac"et\, n. [F. facette, dim. of face face. See {Face}.]
   1. A little face; a small, plane surface; as, the facets of a
      diamond. [Written also {facette}.]
      [1913 Webster]
   2. (Anat.) A smooth circumscribed surface; as, the articular
      facet of a bone.
      [1913 Webster]
   3. (Arch.) The narrow plane surface between flutings of a
      column.
      [1913 Webster]
   4. (Zo["o]l.) One of the numerous small eyes which make up
      the compound eyes of insects and crustaceans.
      [1913 Webster]
Facet \Fac"et\, v. t. [imp. & p. p. {Faceted}; p. pr. & vb. n.
   {Faceting}.]
   To cut facets or small faces upon; as, to facet a diamond.
   [1913 Webster]
\end{lstlisting}}
\end{enumerate}
\item[$\square$] \emph{ reverie }
\begin{enumerate}
\item{
\begin{lstlisting}
(* 
ονειροπόληση, ρεμβασμός *)
\end{lstlisting}}
\item{
\begin{lstlisting}
Reverie \Rev"er*ie\, Revery \Rev"er*y\, n.; pl. {Reveries}. [F.
   r['e]verie, fr. r[^e]ver to dream, rave, be light-headed. Cf.
   {Rave}.]
   1. A loose or irregular train of thought occurring in musing
      or mediation; deep musing; daydream. ``Rapt in nameless
      reveries.'' --Tennyson.
      [1913 Webster]
            When ideas float in our mind without any reflection
            or regard of the understanding, it is that which the
            French call revery, our language has scarce a name
            for it.                               --Locke.
      [1913 Webster]
   2. An extravagant conceit of the fancy; a vision. [R.]
      [1913 Webster]
            There are infinite reveries and numberless
            extravagancies pass through both [wise and foolish
            minds].                               --Addison.
      [1913 Webster]
\end{lstlisting}}
\end{enumerate}
\item[$\square$] \emph{ doppelganger }
\begin{enumerate}
\item{
\begin{lstlisting}
(*  *)
\end{lstlisting}}
\item{
\begin{lstlisting}
Doppelganger \Dop"pel*g["a]ng`er\, n. [G.]
   A spiritual or ghostly double or counterpart; esp., an
   apparitional double of a living person; a cowalker.
   [Webster 1913 Suppl.]
\end{lstlisting}}
\end{enumerate}
\item[$\square$] \emph{ penultimate }
\begin{enumerate}
\item{
\begin{lstlisting}
(*  *)
\end{lstlisting}}
\item{
\begin{lstlisting}
Penultimate \Pe*nul"ti*mate\, a.
   Last but one; as, the penultimate syllable, the last syllable
   but one of a word.
   [1913 Webster]
Penultimate \Pe*nul"ti*mate\, n.
   The penult.
   [1913 Webster]
\end{lstlisting}}
\end{enumerate}
\item[$\square$] \emph{ dilettante }
\begin{enumerate}
\item{
\begin{lstlisting}
(* 
Ουσ. κάποιος που ενδιαφέρεται για κάτι χωρίς όμως να το καταλαβαίνει σε βάθος, ερασιτέχνης, επιφανειακός Επίθ. ερασιτεχνικός, όχι σε βάθος *)
\end{lstlisting}}
\item{
\begin{lstlisting}
Dilettante \Dil`et*tan"te\, n.; pl. {Dilettanti}. [It., prop. p.
   pr. of dillettare to take delight in, fr. L. delectare to
   delight. See {Delight}, v. t.]
   An admirer or lover of the fine arts; popularly, an amateur;
   especially, one who follows an art or a branch of knowledge,
   desultorily, or for amusement only.
   [1913 Webster]
         The true poet is not an eccentric creature, not a mere
         artist living only for art, not a dreamer or a
         dilettante, sipping the nectar of existence, while he
         keeps aloof from its deeper interests.   --J. C.
                                                  Shairp.
   [1913 Webster]
\end{lstlisting}}
\end{enumerate}
\item[$\square$] \emph{ wretched }
\begin{enumerate}
\item{
\begin{lstlisting}
(* 
Επίθ. ελεεινος, αξιολύπητος, τρισάθλιος, // πρόστυχος, πανάθλιος *)
\end{lstlisting}}
\item{
\begin{lstlisting}
Wretched \Wretch"ed\, a.
   1. Very miserable; sunk in, or accompanied by, deep
      affliction or distress, as from want, anxiety, or grief;
      calamitous; woeful; very afflicting. ``To what wretched
      state reserved!'' --Milton.
      [1913 Webster]
            O cruel! Death! to those you are more kind
            Than to the wretched mortals left behind. --Waller.
      [1913 Webster]
            The wretched refuse of your teeming shore . . .
      [1913 Webster]
   2. Worthless; paltry; very poor or mean; miserable; as, a
      wretched poem; a wretched cabin.
      [1913 Webster]
   3. Hatefully contemptible; despicable; wicked. [Obs.]
      ``Wretched ungratefulness.'' --Sir P. Sidney.
      [1913 Webster]
            Nero reigned after this Claudius, of all men
            wretchedest, ready to all manner [of] vices.
                                                  --Capgrave.
      [1913 Webster]
\end{lstlisting}}
\end{enumerate}
\item[$\square$] \emph{ disconsolate }
\begin{enumerate}
\item{
\begin{lstlisting}
(* 
απαρηγόρητος, απελπισμένος *)
\end{lstlisting}}
\item{
\begin{lstlisting}
Disconsolate \Dis*con"so*late\, n.
   Disconsolateness. [Obs.] --Barrow.
   [1913 Webster]
Disconsolate \Dis*con"so*late\, a. [LL. disconsolatus; L. dis- +
   consolatus, p. p. of consolari to console. See {Console}, v.
   t.]
   1. Destitute of consolation; deeply dejected and dispirited;
      hopelessly sad; comfortless; filled with grief; as, a
      bereaved and disconsolate parent.
      [1913 Webster]
            One morn a Peri at the gate
            Of Eden stood disconsolate.           --Moore.
      [1913 Webster]
            The ladies and the knights, no shelter nigh,
            Were dropping wet, disconsolate and wan. --Dryden.
      [1913 Webster]
   2. Inspiring dejection; saddening; cheerless; as, the
      disconsolate darkness of the winter nights. --Ray.
   Syn: Forlorn; melancholy; sorrowful; desolate; woeful;
        hopeless; gloomy. -- {Dis*con"so*late*ly}, adv. --
        {Dis*con"so*late*ness}, n.
        [1913 Webster]
\end{lstlisting}}
\end{enumerate}
\item[$\square$] \emph{ forlorn }
\begin{enumerate}
\item{
\begin{lstlisting}
(* 
Επίθ. εγκαταλελειμένος, απελπισμένος, παραμελημένος, // λυπημένος, μελαγχολικός *)
\end{lstlisting}}
\item{
\begin{lstlisting}
Forlorn \For*lorn"\, a. [OE., p. p. of forlesen to lose utterly,
   AS. forle['o]san (p. p. forloren); pref. for- + le['o]san (in
   comp.) to lose; cf. D. verliezen to lose, G. verlieren, Sw.
   f["o]rlora, Dan. forloren, Goth. fraliusan to lose. See
   {For-}, and {Lorn}, a., {Lose}, v. t.]
   1. Deserted; abandoned; lost.
      [1913 Webster]
            Of fortune and of hope at once forlorn. --Spenser.
      [1913 Webster]
            Some say that ravens foster forlorn children.
                                                  --Shak.
      [1913 Webster]
   2. Destitute; helpless; in pitiful plight; wretched;
      miserable; almost hopeless; desperate.
      [1913 Webster]
            For here forlorn and lost I tread.    --Goldsmith.
      [1913 Webster]
            The condition of the besieged in the mean time was
            forlorn in the extreme.               --Prescott.
      [1913 Webster]
            She cherished the forlorn hope that he was still
            living.                               --Thomson.
      [1913 Webster]
   {A forlorn hope} [D. verloren hoop, prop., a lost band or
      troop; verloren, p. p. of verliezen to lose + hoop band;
      akin to E. heap. See {For-}, and {Heap}.] (Mil.), a body
      of men (called in F. {enfants perdus}, in G. {verlornen
      posten}) selected, usually from volunteers, to attempt a
      breach, scale the wall of a fortress, or perform other
      extraordinarily perilous service; also, a desperate case
      or enterprise.
   Syn: Destitute, lost; abandoned; forsaken; solitary;
        helpless; friendless; hopeless; abject; wretched;
        miserable; pitiable.
        [1913 Webster]
Forlese \For*lese"\, v. t. [p. p. {Forlore}, {Forlorn}.] [OE.
   forlesen. See {Forlorn}.]
   To lose utterly. [Obs.] --haucer.
   [1913 Webster]
Forlorn \For*lorn"\, n.
   1. A lost, forsaken, or solitary person.
      [1913 Webster]
            Forced to live in Scotland a forlorn. --Shak.
      [1913 Webster]
   2. A forlorn hope; a vanguard. [Obs.]
      [1913 Webster]
            Our forlorn of horse marched within a mile of the
            enemy.                                --Oliver
                                                  Cromvell.
      [1913 Webster]
\end{lstlisting}}
\end{enumerate}
\item[$\square$] \emph{ ignoble }
\begin{enumerate}
\item{
\begin{lstlisting}
(* 
Επίθ. ποταπός, χυδαίος, πρόστυχος, αχρείος, μικροπρεπής *)
\end{lstlisting}}
\item{
\begin{lstlisting}
Ignoble \Ig*no"ble\, v. t.
   To make ignoble. [Obs.] --Bacon.
   [1913 Webster]
Ignoble \Ig*no"ble\, a. [L. ignobilis; pref. in- not + nobilis
   noble: cf. F. ignoble. See {In-} not, and {Noble}, a.]
   1. Of low birth or family; not noble; not illustrious;
      plebeian; common; humble.
      [1913 Webster]
            I was not ignoble of descent.         --Shak.
      [1913 Webster]
            Her royal stock graft with ignoble plants. --Shak.
      [1913 Webster]
   2. Not honorable, elevated, or generous; base.
      [1913 Webster]
            'T is but a base, ignoble mind,
            That mounts no higher than a bird can soar. --Shak.
      [1913 Webster]
            Far from the madding crowd's ignoble strife. --Gray.
      [1913 Webster]
   3. (Zo["o]l.) Not a true or noble falcon; -- said of certain
      hawks, as the goshawk.
   Syn: Degenerate; degraded; mean; base; dishonorable;
        reproachful; disgraceful; shameful; scandalous;
        infamous.
        [1913 Webster]
\end{lstlisting}}
\end{enumerate}
\item[$\square$] \emph{ odious }
\begin{enumerate}
\item{
\begin{lstlisting}
(* 
Επιθ. μισητός, απεχθής, αποκρουστικός *)
\end{lstlisting}}
\item{
\begin{lstlisting}
Odious \O"di*ous\, a. [L. odiosus, from odium hatred: cf. F.
   odieux. See {Odium}.]
   1. Hateful; deserving or receiving hatred; as, an odious
      name, system, vice. ``All wickedness will be most
      odious.'' --Sprat.
      [1913 Webster]
            He rendered himself odious to the Parliament.
                                                  --Clarendon.
      [1913 Webster]
   2. Causing or provoking hatred, repugnance, or disgust;
      offensive; disagreeable; repulsive; as, an odious sight;
      an odious smell. --Milton.
      [1913 Webster]
            The odious side of that polity.       --Macaulay.
      [1913 Webster]
   Syn: Hateful; detestable; abominable; disgusting; loathsome;
        invidious; repulsive; forbidding; unpopular.
        [1913 Webster] -- {O"di*ous`ly}. adv. --
        {O"di*ous*ness}, n.
        [1913 Webster]
\end{lstlisting}}
\end{enumerate}
\item[$\square$] \emph{ culminate }
\begin{enumerate}
\item{
\begin{lstlisting}
(* 
Ρημ. μεσουρανώ// φθάνω στο αποκορύφωμα, μεσουρανώ (μτφ.)// καταλήγω, φθάνω ως κατάληξη *)
\end{lstlisting}}
\item{
\begin{lstlisting}
Culminate \Cul"mi*nate\ (k[u^]l"m[i^]*n[=a]t), v. i. [imp. & p.
   p. {Culminated} (-n[=a]`t[e^]d); p. pr. & vb. n.
   {Culminating} (-n[=a]`t[i^]ng.] [L. cuimen top or ridge. See
   {Column}.]
   1. To reach its highest point of altitude; to come to the
      meridian; to be vertical or directly overhead.
      [1913 Webster]
            As when his beams at noon
            Culminate from the equator.           --Milton.
      [1913 Webster]
   2. To reach the highest point, as of rank, size, power,
      numbers, etc.
      [1913 Webster]
            The reptile race culminated in the secondary era.
                                                  --Dana.
      [1913 Webster]
            The house of Burgundy was rapidly culminating.
                                                  --Motley.
      [1913 Webster]
Culminate \Cul"mi*nate\ (k[u^]l"m[i^]*n[asl]t), a.
   Growing upward, as distinguished from a lateral growth; --
   applied to the growth of corals. --Dana.
   [1913 Webster]
\end{lstlisting}}
\end{enumerate}
\item[$\square$] \emph{ osmosis }
\begin{enumerate}
\item{
\begin{lstlisting}
(* 
Ουσ. όσμωση (φυσιολ.) *)
\end{lstlisting}}
\item{
\begin{lstlisting}
osmosis \os*mo"sis\ ([o^]z*m[=o]"s[i^]s), n. [NL., fr. Gr.
   'wsmo`s, equiv. to 'w^sis impulse, fr. 'wqei^n to push.]
   (Chemical Physics)
   (a) The tendency in fluids to mix, or become equably
       diffused, when in contact. It was first observed between
       fluids of differing densities, and as taking place
       through a membrane or an intervening porous structure. An
       older term for the phenomenon was {Osmose}.
   Note: The more rapid flow from the thinner to the thicker
         fluid was then called {endosmosis} (formerly
         {endosmose}), and the opposite, slower current,
         {exosmosis} (formerly {exosmose}). Both are, however,
         results of the same force. Osmosis may be regarded as a
         form of molecular attraction, allied to that of
         adhesion. See also {osmotic pressure}.
   (b) The action produced by this tendency.
       [1913 Webster]
\end{lstlisting}}
\end{enumerate}
\item[$\square$] \emph{ catamite }
\begin{enumerate}
\item{
\begin{lstlisting}
(*  *)
\end{lstlisting}}
\item{
\begin{lstlisting}
Catamite \Cat"a*mite\ (k[a^]t"[.a]*m[imac]t), n. [L. Catamitus,
   an old form of Ganymedes Ganymede, Gr. Ganymh`dhs.]
   A boy kept for unnatural purposes.
   [1913 Webster]
\end{lstlisting}}
\end{enumerate}
\item[$\square$] \emph{ agronomy }
\begin{enumerate}
\item{
\begin{lstlisting}
(*  *)
\end{lstlisting}}
\item{
\begin{lstlisting}
Agronomy \A*gron"o*my\, n. [Gr. ? rural; as a noun, an overseer
   of the public lands; ? field + ? usage, ? to deal out,
   manage: cf. F. agronomie.]
   The management of land; rural economy; agriculture.
   [1913 Webster]
\end{lstlisting}}
\end{enumerate}
\item[$\square$] \emph{ crofter }
\begin{enumerate}
\item{
\begin{lstlisting}
(* 
Ουσ. μικροκαλιεργητής, καλιεργητής κήπου// κάποιος που ζει και εργάζεται σε μικρό υποστατικό *)
\end{lstlisting}}
\item{
\begin{lstlisting}
Crofter \Croft"er\ (-?r), n.
   One who rents and tills a small farm or holding; as, the
   crofters of Scotland.
   [1913 Webster]
\end{lstlisting}}
\end{enumerate}
\item[$\square$] \emph{ harken }
\begin{enumerate}
\item{
\begin{lstlisting}
(*  *)
\end{lstlisting}}
\item{
\begin{lstlisting}
Harken \Hark"en\ (h[aum]rk"'n), v. t. & i.
   To hearken. --Tennyson.
   [1913 Webster]
\end{lstlisting}}
\end{enumerate}
\item[$\square$] \emph{ terrene }
\begin{enumerate}
\item{
\begin{lstlisting}
(* 
Επίθ. γήινος, επίγειος, έδαφος. *)
\end{lstlisting}}
\item{
\begin{lstlisting}
Terrene \Ter*rene"\, n.
   A tureen. [Obs.] --Walpole.
   [1913 Webster]
Terrene \Ter*rene"\, a. [L. terrenus, fr. terra the earth. See
   {Terrace}.]
   1. Of or pertaining to the earth; earthy; as, terrene
      substance. --Holland.
      [1913 Webster]
   2. Earthy; terrestrial.
      [1913 Webster]
            God set before him a mortal and immortal life, a
            nature celestial and terrene.         --Sir W.
                                                  Raleigh.
      [1913 Webster]
            Be true and faithful to the king and his heirs, and
            truth and faith to bear of life and limb, and
            terrene honor.                        --O. Eng. Oath
                                                  of Allegiance,
                                                  quoted by
                                                  Blackstone.
      [1913 Webster]
            Common conceptions of the matters which lie at the
            basis of our terrene experience.      --Hickok.
      [1913 Webster]
Terrene \Ter*rene"\, n. [L. terrenum land, ground: cf. F.
   terrain.]
   1. The earth's surface; the earth. [Poetic]
      [1913 Webster]
            Tenfold the length of this terrene.   --Milton.
      [1913 Webster]
   2. (Surv.) The surface of the ground.
      [1913 Webster]
\end{lstlisting}}
\end{enumerate}
\item[$\square$] \emph{ recidivist }
\begin{enumerate}
\item{
\begin{lstlisting}
(* 
Ουσ. κάποιος που υποτροπιάζει (συν. νομ.) *)
\end{lstlisting}}
\item{
\begin{lstlisting}
Recidivist \Re*cid"i*vist\ (r[-e]*s[i^]d"[i^]*v[i^]st), n.
   One who is recidivous or is characterized by recidivism; an
   incorrigible criminal. -- {Re*cid`i*vis"tic}
   (r[-e]*s[i^]d`[i^]*v[i^]s"t[i^]k), a.
   [Webster 1913 Suppl.]
         The criminal by passion never becomes a recidivist, it
         is the social, not the antisocial, instincts that are
         strong within him, his crime is a solitary event in his
         life.                                    --Havelock
                                                  Ellis.
   [Webster 1913 Suppl.]
\end{lstlisting}}
\end{enumerate}
\item[$\square$] \emph{ malignant }
\begin{enumerate}
\item{
\begin{lstlisting}
(* 
Επιθ.  κακοήθης, κακόβουλος, κακεντρεχής *)
\end{lstlisting}}
\item{
\begin{lstlisting}
Invasive \In*va"sive\, a. [LL. invasivus: cf. F. invasif. See
   {Invade}.]
   1. Tending to invade; characterized by invasion; aggressive.
      ``Invasive war.'' --Hoole.
      [1913 Webster]
   2. (Med.) tending to spread, especially tending to intrude
      into healthy tissue; -- used mostly of tumors. [Narrower
      terms: {malignant}] PJC]
malignant \ma*lig"nant\, a. [L. malignans, -antis, p. pr. of
   malignare, malignari, to do or make maliciously. See
   {Malign}, and cf. {Benignant}.]
   1. Disposed to do harm, inflict suffering, or cause distress;
      actuated by extreme malevolence or enmity; virulently
      inimical; bent on evil; malicious.
      [1913 Webster]
            A malignant and a turbaned Turk.      --Shak.
      [1913 Webster]
   2. Characterized or caused by evil intentions; pernicious.
      ``Malignant care.'' --Macaulay.
      [1913 Webster]
            Some malignant power upon my life.    --Shak.
      [1913 Webster]
            Something deleterious and malignant as his touch.
                                                  --Hawthorne.
      [1913 Webster]
   3. (Med.) Tending to produce death; threatening a fatal
      issue; virulent; as, malignant diphtheria.
      [1913 Webster]
   {Malignant pustule} (Med.), a very contagious disease
      produced by infection of subcutaneous tissues with the
      bacterium {Bacillus anthracis}. It is transmitted to man
      from animals and is characterized by the formation, at the
      point of reception of the infection, of a vesicle or
      pustule which first enlarges and then breaks down into an
      unhealthy ulcer. It is marked by profound exhaustion and
      often fatal. The disease in animals is called {charbon};
      in man it is called {cutaneous anthrax}, and formerly was
      sometimes called simply {anthrax}.
      [1913 Webster +PJC]
\end{lstlisting}}
\end{enumerate}
\item[$\square$] \emph{ simmer }
\begin{enumerate}
\item{
\begin{lstlisting}
(* 
Ουσ. σιγοβράσιμο Ρημ. σιγοβράζω *)
\end{lstlisting}}
\item{
\begin{lstlisting}
Simmer \Sim"mer\, v. i. [imp. & p. p. {Simmered}; p. pr. & vb.
   n. {Simmering}.] [Prov. E. also simper; -- an onomatopoetic
   word.]
   To boil gently, or with a gentle hissing; to begin to boil.
   [1913 Webster]
         I simmer as liquor doth on the fire before it beginneth
         to boil.                                 --Palsgrave.
   [1913 Webster]
Simmer \Sim"mer\, v. t.
   To cause to boil gently; to cook in liquid heated almost or
   just to the boiling point.
   [1913 Webster]
\end{lstlisting}}
\end{enumerate}
\item[$\square$] \emph{ pyroclastic }
\begin{enumerate}
\item{
\begin{lstlisting}
(*  *)
\end{lstlisting}}
\item{
\begin{lstlisting}
Proplastic \Pro*plas"tic\, a.
   Forming a mold.
   [1913 Webster]
\end{lstlisting}}
\end{enumerate}
\item[$\square$] \emph{ tertiary }
\begin{enumerate}
\item{
\begin{lstlisting}
(*  *)
\end{lstlisting}}
\item{
\begin{lstlisting}
Tertiary \Ter"ti*a*ry\, a. [L. tertiarius containing a third
   part, fr. tertius third: cf. F. tertiaire. See {Tierce}.]
   1. Being of the third formation, order, or rank; third; as, a
      tertiary use of a word. --Trench.
      [1913 Webster]
   2. (Chem.) Possessing some quality in the third degree;
      having been subjected to the substitution of three atoms
      or radicals; as, a tertiary alcohol, amine, or salt. Cf.
      {Primary}, and {Secondary}.
      [1913 Webster]
   3. (Geol.) Later than, or subsequent to, the Secondary.
      [1913 Webster]
   4. (Zo["o]l.) Growing on the innermost joint of a bird's
      wing; tertial; -- said of quills.
      [1913 Webster]
   {Tertiary age}. (Geol.) See under {Age}, 8.
   {Tertiary color}, a color produced by the mixture of two
      secondaries. ``The so-called tertiary colors are citrine,
      russet, and olive.'' --Fairholt.
   {Tertiary period}. (Geol.)
      (a) The first period of the age of mammals, or of the
          Cenozoic era.
      (b) The rock formation of that period; -- called also
          {Tertiary formation}. See the Chart of {Geology}.
   {Tertiary syphilis} (Med.), the third and last stage of
      syphilis, in which it invades the bones and internal
      organs.
      [1913 Webster]
Tertiary \Ter"ti*a*ry\, n.; pl. {Tertiaries}.
   1. (R. C. Ch.) A member of the Third Order in any monastic
      system; as, the Franciscan tertiaries; the Dominican
      tertiaries; the Carmelite tertiaries. See {Third Order},
      under {Third}. --Addis & Arnold.
      [1913 Webster]
   2. (Geol.) The Tertiary era, period, or formation.
      [1913 Webster]
   3. (Zo["o]l.) One of the quill feathers which are borne upon
      the basal joint of the wing of a bird. See Illust. of
      {Bird}.
      [1913 Webster]
\end{lstlisting}}
\end{enumerate}
\item[$\square$] \emph{ tractate }
\begin{enumerate}
\item{
\begin{lstlisting}
(* 
Ουσ. διατριβή. *)
\end{lstlisting}}
\item{
\begin{lstlisting}
Tractate \Tract"ate\, n. [L. tractatus a touching, handling,
   treatise. See Tractable, and {Tract} a treatise, {Treaty}.]
   A treatise; a tract; an essay.
   [1913 Webster]
         Agreeing in substance with Augustin's, from whose
         fourteenth Tractate on St. John the words are
         translated.                              --Hare.
   [1913 Webster]
\end{lstlisting}}
\end{enumerate}
\item[$\square$] \emph{ irremediably }
\begin{enumerate}
\item{
\begin{lstlisting}
(* 
ανεπανόρθωτα// αθεράπευτα, ανίατα *)
\end{lstlisting}}
\item{
\begin{lstlisting}
Irremediably \Ir`re*me"di*a*bly\, adv.
   In a manner, or to a degree, that precludes remedy, cure, or
   correction.
   [1913 Webster]
\end{lstlisting}}
\end{enumerate}
\item[$\square$] \emph{ trenchant }
\begin{enumerate}
\item{
\begin{lstlisting}
(* 
Επίθ. δηκτικός, δριμύς// έξυπνος, διεισδυτικός// απερίφραστος *)
\end{lstlisting}}
\item{
\begin{lstlisting}
Trenchant \Trench"ant\, a. [OF. trenchant, F. tranchant, p. pr.
   See {Trench}, v. t.]
   1. Fitted to trench or cut; gutting; sharp. `` Trenchant was
      the blade.'' --Chaucer.
      [1913 Webster]
   2. Fig.: Keen; biting; severe; as, trenchant wit.
      [1913 Webster]
\end{lstlisting}}
\end{enumerate}
\item[$\square$] \emph{ acerbic }
\begin{enumerate}
\item{
\begin{lstlisting}
(* 
Επίθ. καυστικός, δριμύς (για τρόπο γραφής ή έκφρασης) *)
\end{lstlisting}}
\item{
\begin{lstlisting}
Acerbic \A*cerb"ic\, a.
   Sour or severe.
   [1913 Webster]
\end{lstlisting}}
\end{enumerate}
\item[$\square$] \emph{ vehemence }
\begin{enumerate}
\item{
\begin{lstlisting}
(* 
Ουσ. σφοδρότητα, βιαιότητα//  ορμητικότητα *)
\end{lstlisting}}
\item{
\begin{lstlisting}
Vehemence \Ve"he*mence\, n. [L. vehementia: cf. F.
   v['e]h['e]mence.]
   [1913 Webster]
   1. The quality pr state of being vehement; impetuous force;
      impetuosity; violence; fury; as, the vehemence.
      [1913 Webster]
   2. Violent ardor; great heat; animated fervor; as, the
      vehemence of love, anger, or other passions.
      [1913 Webster]
            I . . . tremble at his vehemence of temper.
                                                  --Addison.
      [1913 Webster]
\end{lstlisting}}
\end{enumerate}
\item[$\square$] \emph{ allusory }
\begin{enumerate}
\item{
\begin{lstlisting}
(*  *)
\end{lstlisting}}
\item{
\begin{lstlisting}
Allusory \Al*lu"so*ry\, a.
   Allusive. [R.] --Warburton.
   [1913 Webster]
\end{lstlisting}}
\end{enumerate}
\item[$\square$] \emph{ peregrine }
\begin{enumerate}
\item{
\begin{lstlisting}
(* 
ξένος, αλλοδαπός, μέτοικος// πετρίτης, είδος γερακιού. *)
\end{lstlisting}}
\item{
\begin{lstlisting}
Peregrine \Per"e*grine\, n.
   The peregrine falcon.
   [1913 Webster]
Peregrine \Per"e*grine\, a. [L. peregrinus. See {Pilgrim}.]
   Foreign; not native; extrinsic or from without; exotic.
   [Spelt also {pelegrine}.] ``Peregrine and preternatural
   heat.'' --Bacon.
   [1913 Webster]
   {Peregrine falcon} (Zo["o]l.), a courageous and swift falcon
      ({Falco peregrinus}), remarkable for its wide distribution
      over all the continents. The adult plumage is dark bluish
      ash on the back, nearly black on the head and cheeks,
      white beneath, barred with black below the throat. Called
      also {peregrine hawk}, {duck hawk}, {game hawk}, and
      {great-footed hawk}.
      [1913 Webster]
\end{lstlisting}}
\end{enumerate}
\item[$\square$] \emph{ tract }
\begin{enumerate}
\item{
\begin{lstlisting}
(* 
Ουσ. έκταση, περιοχή//  φυλλάδιο, τεύχος //  χρονικό διαστημα, περίοδος *)
\end{lstlisting}}
\item{
\begin{lstlisting}
Tract \Tract\, n. [Abbrev.fr. tractate.]
   A written discourse or dissertation, generally of short
   extent; a short treatise, especially on practical religion.
   [1913 Webster]
         The church clergy at that time writ the best collection
         of tracts against popery that ever appeared. --Swift.
   [1913 Webster]
   {Tracts for the Times}. See {Tractarian}.
      [1913 Webster]
Tract \Tract\, n. [L. tractus a drawing, train, track, course,
   tract of land, from trahere tractum, to draw. Senses 4 and 5
   are perhaps due to confusion with track. See {Trace},v., and
   cf. {Tratt}.]
   1. Something drawn out or extended; expanse. ``The deep tract
      of hell.'' --Milton.
      [1913 Webster]
   2. A region or quantity of land or water, of indefinite
      extent; an area; as, an unexplored tract of sea.
      [1913 Webster]
            A very high mountain joined to the mainland by a
            narrow tract of earth.                --Addison.
      [1913 Webster]
   3. Traits; features; lineaments. [Obs.]
      [1913 Webster]
            The discovery of a man's self by the tracts of his
            countenance is a great weakness.      --Bacon.
      [1913 Webster]
   4. The footprint of a wild beast. [Obs.] --Dryden.
      [1913 Webster]
   5. Track; trace. [Obs.]
      [1913 Webster]
            Efface all tract of its traduction.   --Sir T.
                                                  Browne.
      [1913 Webster]
            But flies an eagle flight, bold, and forthon,
            Leaving no tract behind.              --Shak.
      [1913 Webster]
   6. Treatment; exposition. [Obs.] --Shak.
      [1913 Webster]
   7. Continuity or extension of anything; as, the tract of
      speech. [Obs.] --Older.
      [1913 Webster]
   8. Continued or protracted duration; length; extent.
      ``Improved by tract of time.'' --Milton.
      [1913 Webster]
   9. (R. C. Ch.) Verses of Scripture sung at Mass, instead of
      the Alleluia, from Septuagesima Sunday till the Saturday
      befor Easter; -- so called because sung tractim, or
      without a break, by one voice, instead of by many as in
      the antiphons.
      [1913 Webster]
   Syn: Region; district; quarter; essay; treatise;
        dissertation.
        [1913 Webster]
Tract \Tract\, v. t.
   To trace out; to track; also, to draw out; to protact. [Obs.]
   --Spenser. --B. Jonson.
   [1913 Webster]
\end{lstlisting}}
\end{enumerate}
\item[$\square$] \emph{ sedentary }
\begin{enumerate}
\item{
\begin{lstlisting}
(* 
Επίθ. καθιστικός, που δεν ασκείται *)
\end{lstlisting}}
\item{
\begin{lstlisting}
Sedentary \Sed"en*ta*ry\, a. [L. sedentarius, fr. sedere to sit:
   cf. F. se['e]dentaire. See {Sedent}.]
   1. Accustomed to sit much or long; as, a sedentary man.
      ``Sedentary, scholastic sophists.'' --Bp. Warburton.
      [1913 Webster]
   2. Characterized by, or requiring, much sitting; as, a
      sedentary employment; a sedentary life.
      [1913 Webster]
            Any education that confined itself to sedentary
            pursuits was essentially imperfect.   --Beaconsfield.
      [1913 Webster]
   3. Inactive; motionless; sluggish; hence, calm; tranquil.
      [R.] ``The sedentary earth.'' --Milton.
      [1913 Webster]
            The soul, considered abstractly from its passions,
            is of a remiss, sedentary nature.     --Spectator.
      [1913 Webster]
   4. Caused by long sitting. [Obs.] ``Sedentary numbness.''
      --Milton.
      [1913 Webster]
   5. (Zo["o]l.) Remaining in one place, especially when firmly
      attached to some object; as, the oyster is a sedentary
      mollusk; the barnacles are sedentary crustaceans.
      [1913 Webster]
   {Sedentary spider} (Zo["o]l.), one of a tribe of spiders
      which rest motionless until their prey is caught in their
      web.
      [1913 Webster]
\end{lstlisting}}
\end{enumerate}
\item[$\square$] \emph{ putrid }
\begin{enumerate}
\item{
\begin{lstlisting}
(* 
Επίθ. σάπιος, χαλασμένος *)
\end{lstlisting}}
\item{
\begin{lstlisting}
Putrid \Pu"trid\, a. [L. putridus, fr. putrere to be rotten, fr.
   puter, or putris, rotten, fr. putere to stink, to be rotten:
   cf. F. putride. See {Pus}, {Foul}, a.]
   1. Tending to decomposition or decay; decomposed; rotten; --
      said of animal or vegetable matter; as, putrid flesh. See
      {Putrefaction}.
      [1913 Webster]
   2. Indicating or proceeding from a decayed state of animal or
      vegetable matter; as, a putrid smell.
      [1913 Webster]
   {Putrid fever} (Med.), typhus fever; -- so called from the
      decomposing and offensive state of the discharges and
      diseased textures of the body.
   {Putrid sore throat} (Med.), a gangrenous inflammation of the
      fauces and pharynx.
      [1913 Webster]
\end{lstlisting}}
\end{enumerate}
\item[$\square$] \emph{ carcinoma }
\begin{enumerate}
\item{
\begin{lstlisting}
(* 
Ουσ. καρκίνωμα, όγκος *)
\end{lstlisting}}
\item{
\begin{lstlisting}
Cancer \Can"cer\, n. [L. cancer, cancri, crab, ulcer, a sign of
   the zodiac; akin to Gr. karki`nos, Skr. karka[.t]a crab, and
   prob. Skr. karkara hard, the crab being named from its hard
   shell. Cf. {Canner}, {Chancre}.]
   1. (Zo["o]l.) A genus of decapod Crustacea, including some of
      the most common shore crabs of Europe and North America,
      as the rock crab, Jonah crab, etc. See {Crab}.
      [1913 Webster]
   2. (Astron.)
      (a) The fourth of the twelve signs of the zodiac. The
          first point is the northern limit of the sun's course
          in summer; hence, the sign of the summer solstice. See
          {Tropic}.
      (b) A northern constellation between Gemini and Leo.
          [1913 Webster]
   3. (Med.) Formerly, any malignant growth, esp. one attended
      with great pain and ulceration, with cachexia and
      progressive emaciation. It was so called, perhaps, from
      the great veins which surround it, compared by the
      ancients to the claws of a crab. The term is now
      restricted to such a growth made up of aggregations of
      epithelial cells, either without support or embedded in
      the meshes of a trabecular framework.
      [1913 Webster]
   Note: Four kinds of cancers are recognized: (1) {Epithelial
         cancer, or Epithelioma}, in which there is no
         trabecular framework. See {Epithelioma}. (2) {Scirrhous
         cancer, or Hard cancer}, in which the framework
         predominates, and the tumor is of hard consistence and
         slow growth. (3) {Encephaloid cancer}, {Medullary
         cancer}, or {Soft cancer}, in which the cellular
         element predominates, and the tumor is soft, grows
         rapidy, and often ulcerates. (4) {Colloid cancer}, in
         which the cancerous structure becomes gelatinous. The
         last three varieties are also called {carcinoma}.
         [1913 Webster]
   {Cancer cells}, cells once believed to be peculiar to
      cancers, but now know to be epithelial cells differing in
      no respect from those found elsewhere in the body, and
      distinguished only by peculiarity of location and
      grouping.
   {Cancer root} (Bot.), the name of several low plants, mostly
      parasitic on roots, as the beech drops, the squawroot,
      etc.
   {Tropic of Cancer}. See {Tropic}.
      [1913 Webster]
carcinoma \car`ci*no"ma\ (k[aum]r`s[i^]*n[=o]"m[.a]), n. [L.,
   fr. Gr. karki`nwma, fr. karki`nos crab, cancer. See {-oma}.]
   (Med.)
   A form of malignant cancer arising from epithelial tissue.
   The term was earlier applied to all forms of cancer, or to
   certain non-malignant forms. It is contrasted with {sarcoma},
   a malignant form of cancer arising from connective tissue.
   See {Cancer}. --Dunglison. --Stedman.
   [1913 Webster +PJC]
\end{lstlisting}}
\end{enumerate}
\item[$\square$] \emph{ obscene }
\begin{enumerate}
\item{
\begin{lstlisting}
(* 
Επίθ. πρόστυχος, αισχρός, ανάρμοστος, χυδαίος *)
\end{lstlisting}}
\item{
\begin{lstlisting}
Obscene \Ob*scene"\, a. [L. obscenus, obscaenus, obscoenus, ill
   looking, filthy, obscene: cf. F. obsc['e]ne.]
   [1913 Webster]
   1. Offensive to chastity or modesty; expressing or presenting
      to the mind or view something which delicacy, purity, and
      decency forbid to be exposed; impure; as, obscene
      language; obscene pictures.
      [1913 Webster]
            Words that were once chaste, by frequent use grew
            obscene and uncleanly.                --I. Watts.
      [1913 Webster]
   2. Foul; fifthy; disgusting.
      [1913 Webster]
            A girdle foul with grease binds his obscene attire.
                                                  --Dryden
                                                  (Aeneid, vi.
                                                  417).
      [1913 Webster]
   3. Inauspicious; ill-omened. [R.] [A Latinism]
      [1913 Webster]
            At the cheerful light,
            The groaning ghosts and birds obscene take flight.
                                                  --Dryden.
      [1913 Webster]
   Syn: Impure; immodest; indecent; unchaste; lewd.
        [1913 Webster] -- {Ob*scene"ly}, adv. --
        {Ob*scene"ness}, n.
        [1913 Webster]
\end{lstlisting}}
\end{enumerate}
\item[$\square$] \emph{ recant }
\begin{enumerate}
\item{
\begin{lstlisting}
(* 
Ουσ. ανακαλώ (κάτι που έχω πεί πριν), αναιρώ, παίρνω πίσω,// αποκηρύσσω *)
\end{lstlisting}}
\item{
\begin{lstlisting}
Recant \Re*cant"\, v. i.
   To revoke a declaration or proposition; to unsay what has
   been said; to retract; as, convince me that I am wrong, and I
   will recant. --Dryden.
   [1913 Webster]
Recant \Re*cant"\ (r[-e]*k[a^]nt"), v. t. [imp. & p. p.
   {Recanted}; p. pr. & vb. n. {Recanting}.] [L. recantare,
   recantatum, to recall, recant; pref. re- re- + cantare to
   sing, to sound. See 3d {Cant}, {Chant}.]
   To withdraw or repudiate formally and publicly (opinions
   formerly expressed); to contradict, as a former declaration;
   to take back openly; to retract; to recall.
   [1913 Webster]
         How soon . . . ease would recant
         Vows made in pain, as violent and void!  --Milton.
   [1913 Webster]
   Syn: To retract; recall; revoke; abjure; disown; disavow. See
        {Renounce}.
        [1913 Webster]
\end{lstlisting}}
\end{enumerate}
\item[$\square$] \emph{ mnemonic }
\begin{enumerate}
\item{
\begin{lstlisting}
(* 
Επίθ. μνημονικός , // σχεδιασμένος ωστε να βοηθά την μνήμη Ουσ. οτιδήποτε βοηθά την μνήμη (πχ. λέξη κτλ)// μνημονική *)
\end{lstlisting}}
\item{
\begin{lstlisting}
mnemonic \mnemonic\ n.
   1. Something used to assist the memory, as an easily
      remembered acronym or verse.
      [WordNet 1.5]
   2. An abbreviated word that resembles the full word, used so
      as to be easily recognized; as, the CIDE uses ... tags as
      mnemnonics for an italicised word or field.
      [PJC]
   Note: In basic organic chemistry class, one may learn the
         mnenomic ``Oh my, such good apple pie'' to help
         remember the names of the dicarboxylic acids in
         increasing order of length, namely: oxalic, malonic,
         succinic, glutaric, adipic, and pimelic acids. (From L.
         Fieser's Organic Chemistry text).
\end{lstlisting}}
\end{enumerate}
\item[$\square$] \emph{ morose }
\begin{enumerate}
\item{
\begin{lstlisting}
(* 
Επίθ. δύσθυμος, σκυθρωπός, κακότροπος *)
\end{lstlisting}}
\item{
\begin{lstlisting}
Morose \Mo*rose"\ (m[-o]*r[=o]s"), a. [L. morosus, prop.,
   excessively addicted to any particular way or habit, fr. mos,
   moris, manner, habit, way of life: cf. F. morose.]
   1. Of a sour temper; sullen and austere; ill-humored; severe.
      ``A morose and affected taciturnity.'' --I. Watts.
      [1913 Webster]
   2. Lascivious; brooding over evil thoughts. [Obs.]
      [1913 Webster]
   Syn: Sullen; gruff; severe; austere; gloomy; crabbed; crusty;
        churlish; surly; ill-humored.
        [1913 Webster]
\end{lstlisting}}
\end{enumerate}
\item[$\square$] \emph{ lugubrious }
\begin{enumerate}
\item{
\begin{lstlisting}
(* 
Επίθ. πένθιμος, μελαγχολικός *)
\end{lstlisting}}
\item{
\begin{lstlisting}
Lugubrious \Lu*gu"bri*ous\, a. [L. lugubris, fr. lugere to
   mourn; cf. Gr. lygro`s sad, Skr. ruj to break.]
   Mournful; indicating sorrow, often ridiculously or feignedly;
   doleful; woful; pitiable; as, a whining tone and a lugubrious
   look.
   [1913 Webster]
         Crossbones, scythes, hourglasses, and other lugubrious
         emblems of mortality.                    --Hawthorne.
   -- {Lu*gu"bri*ous*ly}, adv. -- {Lu*gu"bri*ous*ness}, n.
   [1913 Webster]
\end{lstlisting}}
\end{enumerate}

\item[$\square$] \emph{ despondent }
\begin{enumerate}
\item{
\begin{lstlisting}
(* 
Επίθ. απελπισμένος, αποθαρρυμένος, δυστυχισμένος *)
\end{lstlisting}}
\item{
\begin{lstlisting}
Despondent \De*spond"ent\, a. [L. despondens, -entis, p. pr. of
   despond[=e]re.]
   Marked by despondence; given to despondence; low-spirited;
   as, a despondent manner; a despondent prisoner. --
   {De*spond"ent*ly}, adv.
   [1913 Webster]
\end{lstlisting}}
\end{enumerate}
\item[$\square$] \emph{ woebegone }
\begin{enumerate}
\item{
\begin{lstlisting}
(* 
Επιθ. δυστυχής, αξιοθρήνητος, περίλυπος, θλιμμένος *)
\end{lstlisting}}
\item{
\begin{lstlisting}
Woe-begone \Woe"-be*gone`\, a. [OE. wo begon. See {Woe}, and
   {Begone}, p. p.]
   Beset or overwhelmed with woe; immersed in grief or sorrow;
   woeful. --Chaucer.
   [1913 Webster]
         So woe-begone was he with pains of love. --Fairfax.
   [1913 Webster]
\end{lstlisting}}
\end{enumerate}
\item[$\square$] \emph{ curtail }
\begin{enumerate}
\item{
\begin{lstlisting}
(* 
Ρημ. περικόπτω, περιορίζω, περιστέλλω, μειώνω *)
\end{lstlisting}}
\item{
\begin{lstlisting}
Curtail \Cur"tail\ (k?r"t?l), n.
   The scroll termination of any architectural member, as of a
   step, etc.
   [1913 Webster]
Curtail \Cur*tail"\ (k[u^]r*t[=a]l"), v. t. [imp. & p. p.
   {Curtailed} (-t[=a]ld"); p. pr. & vb. n. {Curtailing}.] [See
   {Curtal}.]
   To cut off the end or tail, or any part, of; to shorten; to
   abridge; to diminish; to reduce.
   [1913 Webster]
         I, that am curtailed of this fair proportion. --Shak.
   [1913 Webster]
         Our incomes have been curtailed; his salary has been
         doubled.                                 --Macaulay.
   [1913 Webster]
\end{lstlisting}}
\end{enumerate}

\item[$\square$] \emph{ conglomerate }
\begin{enumerate}
\item{
\begin{lstlisting}
(* 
Ρημ. συμπυκνώνω, //  συμπιέζομαι, συσσωρεύομαι, συσσωρεύω Ουσ. σύμφυρμα, συσσώρευση-συμπύκνωση πολλών διαφορετικών υλικών *)
\end{lstlisting}}
\item{
\begin{lstlisting}
Conglomerate \Con*glom"er*ate\, a. [L. conglomeratus, p. p. of
   conglomerare to roll together; con- + glomerare to wind into
   a ball. See {Glomerate}.]
   1. Gathered into a ball or a mass; collected together;
      concentrated; as, conglomerate rays of light.
      [1913 Webster]
            Beams of light when they are multiplied and
            conglomerate.                         --Bacon.
      [1913 Webster]
            Fluids are separated in the liver and the other
            conglobate and conglomerate glands.   --Cheyne.
      [1913 Webster]
   2. (Bot.) Closely crowded together; densly clustered; as,
      conglomerate flowers. --Gray.
      [1913 Webster]
   3. (Geol.) Composed of stones, pebbles, or fragments of
      rocks, cemented together.
      [1913 Webster]
Conglomerate \Con*glom"er*ate\, n.
   1. That which is heaped together in a mass or conpacted from
      various sources; a mass formed of fragments; collection;
      accumulation.
      [1913 Webster]
            A conglomerate of marvelous anecdotes, marvelously
            heaped together.                      --Trench.
      [1913 Webster]
   2. (Geol.) A rock, composed or rounded fragments of stone
      cemented together by another mineral substance, either
      calcareous, siliceous, or argillaceous; pudding stone; --
      opposed to agglomerate. See {Breccia}.
      [1913 Webster]
            A conglomerate, therefore, is simply gravel bound
            together by a cement.                 --Lyell.
      [1913 Webster]
Conglomerate \Con*glom"er*ate\, v. t. [imp. & p. p.
   {Conglomerated}; p. pr. & vb. n. {Conglomerating}.]
   To gather into a ball or round body; to collect into a mass.
   [1913 Webster]
\end{lstlisting}}
\end{enumerate}
\item[$\square$] \emph{ hermeneutic }
\begin{enumerate}
\item{
\begin{lstlisting}
(* 
Επίθ. ερμηνευτικός (για τα άγια κείμενα) *)
\end{lstlisting}}
\item{
\begin{lstlisting}
Hermeneutic \Her`me*neu"tic\, Hermeneutical \Her`me*neu"tic*al\,
   a. [Gr. ?, fr. ? to interpret: cf. F. herm['e]neutique.]
   Unfolding the signification; of or pertaining to
   interpretation; exegetical; explanatory; as, hermeneutic
   theology, or the art of expounding the Scriptures; a
   hermeneutic phrase.
   [1913 Webster]
\end{lstlisting}}
\end{enumerate}
\item[$\square$] \emph{ supine }
\begin{enumerate}
\item{
\begin{lstlisting}
(* 
Επίθ. ξαπλωμένος, ύπτιος//  αδρανής, ράθυμος, άτονος *)
\end{lstlisting}}
\item{
\begin{lstlisting}
Supine \Su*pine"\, a. [L. supinus, akin to sub under, super
   above. Cf. {Sub-}, {Super-}.]
   1. Lying on the back, or with the face upward; -- opposed to
      prone.
      [1913 Webster]
   2. Leaning backward, or inclining with exposure to the sun;
      sloping; inclined.
      [1913 Webster]
            If the vine
            On rising ground be placed, or hills supine.
                                                  --Dryden.
      [1913 Webster]
   3. Negligent; heedless; indolent; listless.
      [1913 Webster]
            He became pusillanimous and supine, and openly
            exposed to any temptation.            --Woodward.
      [1913 Webster]
   Syn: Negligent; heedless; indolent; thoughtless; inattentive;
        listless; careless; drowsy.
        [1913 Webster] -- {Su*pine"ly}, adv. -- {Su*pine"ness},
        n.
        [1913 Webster]
Supine \Su"pine\, n. [L. supinum (sc. verbum), from supinus bent
   or thrown backward, perhaps so called because, although
   furnished with substantive case endings, it rests or falls
   back, as it were, on the verb: cf. F. supin.] (Lat. Gram.)
   A verbal noun; or (according to C.F.Becker), a case of the
   infinitive mood ending in -um and -u, that in -um being
   sometimes called the former supine, and that in -u the latter
   supine.
   [1913 Webster]
\end{lstlisting}}
\end{enumerate}
\item[$\square$] \emph{ curvaceous }
\begin{enumerate}
\item{
\begin{lstlisting}
(*  *)
\end{lstlisting}}
\item{
\begin{lstlisting}
curvaceous \curvaceous\ adj.
   having a pronounced womanly shape; having a slender waist
   with prominent breasts and hips. [chiefly dialect]
   Syn: bosomy, buxom, full-bosomed, sonsie, sonsy, voluptuous.
        [WordNet 1.5 +PJC]
\end{lstlisting}}
\end{enumerate}
\item[$\square$] \emph{ vicarious }
\begin{enumerate}
\item{
\begin{lstlisting}
(*  *)
\end{lstlisting}}
\item{
\begin{lstlisting}
Vicarious \Vi*ca"ri*ous\, a. [L. vicarius, from vicis change,
   alternation, turn, the position, place, or office of one
   person as assumed by another; akin to Gr. ? to yield, give
   way, G. wechsel a change, and probably also to E. weak. See
   {Weak}, and cf. {Vice}, prep.]
   1. Of or pertaining to a vicar, substitute, or deputy;
      deputed; delegated; as, vicarious power or authority.
      [1913 Webster]
   2. Acting of suffering for another; as, a vicarious agent or
      officer.
      [1913 Webster]
            The soul in the body is but a subordinate efficient,
            and vicarious . . . in the hands of the Almighty.
                                                  --Sir M. Hale.
      [1913 Webster]
   3. Performed of suffered in the place of another;
      substituted; as, a vicarious sacrifice; vicarious
      punishment.
      [1913 Webster]
            The vicarious work of the Great Deliverer. --I.
                                                  Taylor.
      [1913 Webster]
   4. (Med.) Acting as a substitute; -- said of abnormal action
      which replaces a suppressed normal function; as, vicarious
      hemorrhage replacing menstruation.
      [1913 Webster]
\end{lstlisting}}
\end{enumerate}
\item[$\square$] \emph{ furlough }
\begin{enumerate}
\item{
\begin{lstlisting}
(* 
άδεια απουσίας (στρατιώτη, έργάτη κτλ)// δίνω άδεια απουσίας *)
\end{lstlisting}}
\item{
\begin{lstlisting}
Furlough \Fur"lough\, v. t. [imp. & p. p. {Furloughed}; p. pr. &
   vb. n. {Furloughing}.] (Mil.)
   To furnish with a furlough; to grant leave of absence to, as
   to an officer or soldier.
Furlough \Fur"lough\, n. [Prob. fr. D. verlof, fr. a prefix akin
   to E. for + the root of E. lief, and akin to Dan. forlov, Sw.
   f["o]rlof, G. verlaub permission. See {Life}, a.] (Mil.)
   Leave of absence; especially, leave given to an officer or
   soldier to be absent from service for a certain time; also,
   the document granting leave of absence.
   [1913 Webster]
\end{lstlisting}}
\end{enumerate}
\item[$\square$] \emph{ nulliparity }
\begin{enumerate}
\item{
\begin{lstlisting}
(*  *)
\end{lstlisting}}
\item{
\begin{lstlisting}

\end{lstlisting}}
\end{enumerate}
\item[$\square$] \emph{ inchoate }
\begin{enumerate}
\item{
\begin{lstlisting}
(* 
Επιθ. που μόλις άρχισε να διαμορφώνεται, // ατελής, ανοργάνωτος *)
\end{lstlisting}}
\item{
\begin{lstlisting}
Inchoate \In"cho*ate\, a. [L. inchoatus, better incohatus, p. p.
   of incohare to begin.]
   Recently, or just, begun; beginning; partially but not fully
   in existence or operation; existing in its elements;
   incomplete. -- {In"cho*ate*ly}, adv.
   [1913 Webster]
         Neither a substance perfect, nor a substance inchoate.
                                                  --Raleigh.
   [1913 Webster]
Inchoate \In"cho*ate\, v. t.
   To begin. [Obs.] --Dr. H. More.
   [1913 Webster]
\end{lstlisting}}
\end{enumerate}
\item[$\square$] \emph{ consortium }
\begin{enumerate}
\item{
\begin{lstlisting}
(* 
Ουσ. κοινοπραξία, κονσόρτσιο, συνεταιρισμός *)
\end{lstlisting}}
\item{
\begin{lstlisting}
Your choice[-1 to abort]: 1
Consortion \Con*sor"tion\ (k[o^]n*s[^o]r"sh[u^]n), n. [L.
   consortio.]
   Fellowship; association; companionship. [Obs.] --Sir T.
   Browne.
   [1913 Webster]
\end{lstlisting}}
\end{enumerate}
\item[$\square$] \emph{ magnanimity }
\begin{enumerate}
\item{
\begin{lstlisting}
(* 
Ουσ. μεγαλοψυχία *)
\end{lstlisting}}
\item{
\begin{lstlisting}
Magnanimity \Mag`na*nim"i*ty\, n. [F. magnanimit['e], L.
   magnanimitas.]
   The quality of being magnanimous; greatness of mind;
   elevation or dignity of soul; that quality or combination of
   qualities, in character, which enables one to encounter
   danger and trouble with tranquility and firmness, to disdain
   injustice, meanness and revenge, and to act and sacrifice for
   noble objects.
   [1913 Webster]
\end{lstlisting}}
\end{enumerate}
\item[$\square$] \emph{ impudence }
\begin{enumerate}
\item{
\begin{lstlisting}
(* 
αναίδεια , θράσος *)
\end{lstlisting}}
\item{
\begin{lstlisting}
Impudence \Im"pu*dence\ ([i^]m"p[-u]*dens), n. [L. impudentia:
   cf. F. impudence. See {Impudent}.]
   The quality of being impudent; assurance, accompanied with a
   disregard of the presence or opinions of others;
   shamelessness; forwardness; lack of modesty.
   [1913 Webster]
         Clear truths that their own evidence forces us to
         admit, or common experience makes it impudence to deny.
                                                  --Locke.
   [1913 Webster]
         Where pride and impudence (in fashion knit)
         Usurp the chair of wit.                  --B. Jonson.
   Syn: Shamelessness; audacity; insolence; effrontery;
        sauciness; impertinence; pertness; rudeness.
   Usage: {Impudence}, {Effrontery}, {Sauciness}. Impudence
          refers more especially to the feelings as manifested
          in action. Effrontery applies to some gross and public
          exhibition of shamelessness. Sauciness refers to a
          sudden pert outbreak of impudence, especially from an
          inferior. Impudence is an unblushing kind of
          impertinence, and may be manifested in words, tones,
          gestures, looks, etc. Effrontery rises still higher,
          and shows a total or shameless disregard of duty or
          decorum under the circumstances of the case. Sauciness
          discovers itself toward particular individuals, in
          certain relations; as in the case of servants who are
          saucy to their masters, or children who are saucy to
          their teachers. See {Impertinent}, and {Insolent}.
          [1913 Webster]
\end{lstlisting}}
\end{enumerate}
\item[$\square$] \emph{ frivolous }
\begin{enumerate}
\item{
\begin{lstlisting}
(* 
Επιθ. επιπόλαιος, ελαφρός, τραλαλά *)
\end{lstlisting}}
\item{
\begin{lstlisting}
Frivolous \Friv"o*lous\, a. [L. frivolus; prob. akin to friare
   to rub, crumble, E. friable: cf. F. frivole.]
   [1913 Webster]
   1. Of little weight or importance; not worth notice; slight;
      as, a frivolous argument. --Swift.
      [1913 Webster]
   2. Given to trifling; marked with unbecoming levity; silly;
      interested especially in trifling matters.
      [1913 Webster]
            His personal tastes were low and frivolous.
                                                  --Macaulay.
   Syn: Trifling; trivial; slight; petty; worthless. --
        {Friv"o*lous*ly}, adv. -- {Friv"o*lous*ness}, n.
        [1913 Webster]
\end{lstlisting}}
\end{enumerate}
\item[$\square$] \emph{ affluence }
\begin{enumerate}
\item{
\begin{lstlisting}
(* 
Ουσ. αφθονία, πλούτος *)
\end{lstlisting}}
\item{
\begin{lstlisting}
Affluence \Af"flu*ence\, n. [F. affluence, L. affluentia, fr.
   affluens, p. pr. of affluere to flow to; ad + fluere to flow.
   See {Flux}.]
   1. A flowing to or towards; a concourse; an influx.
      [1913 Webster]
            The affluence of young nobles from hence into Spain.
                                                  --Wotton.
      [1913 Webster]
            There is an unusual affluence of strangers this
            year.                                 --Carlyle.
      [1913 Webster]
   2. An abundant supply, as of thought, words, feelings, etc.;
      profusion; also, abundance of property; wealth.
      [1913 Webster]
            And old age of elegance, affluence, and ease.
                                                  --Coldsmith.
      [1913 Webster]
   Syn: Abundance; riches; profusion; exuberance; plenty;
        wealth; opulence.
        [1913 Webster]
\end{lstlisting}}
\end{enumerate}
\item[$\square$] \emph{ implore }
\begin{enumerate}
\item{
\begin{lstlisting}
(* 
ικετεύω, εκλιπαρώ *)
\end{lstlisting}}
\item{
\begin{lstlisting}
Implore \Im*plore"\, v. i.
   To entreat; to beg; to prey.
   [1913 Webster]
Implore \Im*plore"\, n.
   Imploration. [Obs.] --Spencer.
   [1913 Webster]
Implore \Im*plore"\, v. t. [imp. & p. p. {Implored}; p. pr. &
   vb. n. {Imploring}.] [L. implorare; pref. im- in + plorare to
   cry aloud. See {Deplore}.]
   To call upon, or for, in supplication; to beseech; to pray
   to, or for, earnestly; to petition with urgency; to entreat;
   to beg; -- followed directly by the word expressing the thing
   sought, or the person from whom it is sought.
   [1913 Webster]
         Imploring all the gods that reign above. --Pope.
   [1913 Webster]
         I kneel, and then implore her blessing.  --Shak.
   Syn: To beseech; supplicate; crave; entreat; beg; solicit;
        petition; prey; request; adjure. See {Beseech}.
        [1913 Webster]
\end{lstlisting}}
\end{enumerate}
\end{itemize}
\end{document}
