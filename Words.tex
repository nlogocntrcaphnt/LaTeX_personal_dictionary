\documentclass{article}

\usepackage{amssymb}
\usepackage[utf8]{inputenc}
\usepackage[greek,english,russian]{babel}
\usepackage{alphabeta}
\usepackage{titling}
\usepackage[hidelinks]{hyperref}
\usepackage{listings}
\setlength{\droptitle}{-2cm}
\usepackage[a4paper,top=3cm,bottom=2cm,left=2.75cm,right=2.75cm,marginparwidth=1.75cm]{geometry}

\author{Θад Капэтанакис\\ {\small Computer Science Department, University of Athens}\\ {\tiny https://github.com/nlogocntrcaphnt}}
\title{My very first {\LaTeX} document: The Dictionary of Shame}
\date{\textit{Anno Domini, MMXIX}}

\lstset{
basicstyle=\small\rmfamily,
columns=fullflexible,
breaklines=true,
escapeinside={(*}{*)}
}

\begin{document}
	\maketitle
%	\renewcommand{\labelitemi}{$\blacksquare$}
%	\renewcommand\labelitemii{$\square$}
	\renewcommand{\labelenumi}{\Roman{enumi}}
	\begin{itemize}

\item[$\square$] \emph{ proliferation }
\begin{enumerate}
\item{
\begin{lstlisting}
Proliferation \Pro*lif`er*a"tion\, n.
   [1913 Webster]
   1. (Biol.) The continuous development of cells in tissue
      formation; cell formation. --Virchow.
      [1913 Webster]
   2. (Zo["o]l.) The production of numerous zooids by budding,
      especially when buds arise from other buds in succession.
      [1913 Webster]
\end{lstlisting}}
\end{enumerate}
\item[$\square$] \emph{ conjecture }
\begin{enumerate}
\item{
\begin{lstlisting}
Conjecture \Con*jec"ture\, v. t. [imp. & p. p. {Conjectured}; p.
   pr. & vb. n. {Conjecturing}.] [Cf. F. conjecturer. Cf.
   {Conject}.]
   To arrive at by conjecture; to infer on slight evidence; to
   surmise; to guess; to form, at random, opinions concerning.
   [1913 Webster]
         Human reason can then, at the best, but conjecture what
         will be.                                 --South.
   [1913 Webster]
Conjecture \Con*jec"ture\, v. i.
   To make conjectures; to surmise; to guess; to infer; to form
   an opinion; to imagine.
   [1913 Webster]
Conjecture \Con*jec"ture\ (; 135?), n. [L. conjectura, fr.
   conjicere, conjectum, to throw together, infer, conjecture;
   con- + jacere to throw: cf. F. conjecturer. See {Jet} a
   shooting forth.]
   An opinion, or judgment, formed on defective or presumptive
   evidence; probable inference; surmise; guess; suspicion.
   [1913 Webster]
         He [Herodotus] would thus have corrected his first
         loose conjecture by a real study of nature. --Whewell.
   [1913 Webster]
         Conjectures, fancies, built on nothing firm. --Milton.
   [1913 Webster]
\end{lstlisting}}
\end{enumerate}
\item[$\square$] \emph{ sardonic }
\begin{enumerate}
\item{
\begin{lstlisting}
Sardonic \Sar*don"ic\, a.
   Of, pertaining to, or resembling, a kind of linen made at
   Colchis.
   [1913 Webster]
Sardonic \Sar*don"ic\, a. [F. sardonique, L. sardonius, Gr. ?,
   ?, perhaps fr. ? to grin like a dog, or from a certain plant
   of Sardinia, Gr. ?, which was said to screw up the face of
   the eater.]
   Forced; unnatural; insincere; hence, derisive, mocking,
   malignant, or bitterly sarcastic; -- applied only to a laugh,
   smile, or some facial semblance of gayety.
   [1913 Webster]
         Where strained, sardonic smiles are glozing still,
         And grief is forced to laugh against her will. --Sir H.
                                                  Wotton.
   [1913 Webster]
         The scornful, ferocious, sardonic grin of a bloody
         ruffian.                                 --Burke.
   [1913 Webster]
   {Sardonic grin} or {Sardonic laugh}, an old medical term for
      a spasmodic affection of the muscles of the face, giving
      it an appearance of laughter.
      [1913 Webster]
\end{lstlisting}}
\end{enumerate}
\item[$\square$] \emph{ interstice }
\begin{enumerate}
\item{
\begin{lstlisting}
Interstice \In*ter"stice\ (?; 277), n.; pl. {Interstices}. [L.
   interstitium a pause, interval; inter between + sistere to
   set, fr. stare to stand: cf. F. interstice. See {Stand}.]
   [1913 Webster]
   1. That which intervenes between one thing and another;
      especially, a space between things closely set, or between
      the parts which compose a body; a narrow chink; a crack; a
      crevice; a hole; an interval; as, the interstices of a
      wall.
      [1913 Webster]
   2. An interval of time; specifically (R. C. Ch.), in the
      plural, the intervals which the canon law requires between
      the reception of the various degrees of orders.
      [1913 Webster]
            Nonobservance of the interstices . . . is a sin.
                                                  --Addis &
                                                  Arnold.
      [1913 Webster]
\end{lstlisting}}
\end{enumerate}
\item[$\square$] \emph{ effervesce }
\begin{enumerate}
\item{
\begin{lstlisting}
Effervesce \Ef`fer*vesce"\, v. i. [imp. & p. p. {Effervesced};
   p. pr. & vb. n. {Effervescing}.] [L. effervescere; ex +
   fervescere to begin boiling, incho., fr. fervere to boil. See
   {Fervent}.]
   1. To be in a state of natural ebullition; to bubble and
      hiss, as fermenting liquors, or any fluid, when some part
      escapes in a gaseous form.
      [1913 Webster]
   2. To exhibit, in lively natural expression, feelings that
      can not be repressed or concealed; as, to effervesce with
      joy or merriment.
\end{lstlisting}}
\end{enumerate}
\item[$\square$] \emph{ ebullition }
\begin{enumerate}
\item{
\begin{lstlisting}
Ebullition \Eb`ul*li"tion\, n. [F. ['e]bullition, L. ebullitio,
   fr. ebullire. See {Ebullient}.]
   1. A boiling or bubbling up of a liquid; the motion produced
      in a liquid by its rapid conversion into vapor.
      [1913 Webster]
   2. Effervescence occasioned by fermentation or by any other
      process which causes the liberation of a gas or an
      a["e]riform fluid, as in the mixture of an acid with a
      carbonated alkali. [Formerly written {bullition}.]
      [1913 Webster]
   3. A sudden burst or violent display; an outburst; as, an
      ebullition of anger or ill temper.
      [1913 Webster]
\end{lstlisting}}
\end{enumerate}
\item[$\square$] \emph{ serendipity }
\begin{enumerate}
\item{
\begin{lstlisting}

\end{lstlisting}}
\end{enumerate}
\item[$\square$] \emph{ impetus }
\begin{enumerate}
\item{
\begin{lstlisting}
Impetus \Im"pe*tus\ ([i^]m"p[-e]*t[u^]s), n. [L., fr. impetere
   to rush upon, attack; pref. im- in + petere to fall upon,
   seek. See {Petition}.]
   1. A property possessed by a moving body in virtue of its
      weight and its motion; the force with which any body is
      driven or impelled; momentum.
      [1913 Webster]
   Note: Momentum is the technical term, impetus its popular
         equivalent, yet differing from it as applied commonly
         to bodies moving or moved suddenly or violently, and
         indicating the origin and intensity of the motion,
         rather than its quantity or effectiveness.
         [1913 Webster]
   2. Fig.: Impulse; incentive; stimulus; vigor; force; as, the
      President's strong recommendation provided the impetus
      needed to pass the campaign reform bill. --Buckle.
      [1913 Webster +PJC]
   3. (Gun.) The altitude through which a heavy body must fall
      to acquire a velocity equal to that with which a ball is
      discharged from a piece.
      [1913 Webster]
\end{lstlisting}}
\end{enumerate}
\item[$\square$] \emph{ phthisis }
\begin{enumerate}
\item{
\begin{lstlisting}
Phthisis \Phthi"sis\, n. [L., fr. Gr. ?, fr. ? to pass or waste
   away: cf. F. phthisie.] (Med.)
   A wasting or consumption of the tissues. The term was
   formerly applied to many wasting diseases, but is now usually
   restricted to pulmonary phthisis, or consumption. See
   {Consumption}.
   [1913 Webster]
   {Fibroid phthisis}. See under {Fibroid}.
      [1913 Webster]
\end{lstlisting}}
\end{enumerate}
\item[$\square$] \emph{ mendacious }
\begin{enumerate}
\item{
\begin{lstlisting}
Mendacious \Men*da"cious\, a. [L. mendax, -acis, lying, cf.
   mentiri to lie.]
   1. Given to deception or falsehood; lying; as, a mendacious
      person.
      [1913 Webster]
   2. False; counterfeit; containing falsehood; as, a mendacious
      statement.
      [1913 Webster] -- {Men*da"cious*ly}, adv. --
      {Men*da"cious*ness}, n.
      [1913 Webster]
\end{lstlisting}}
\end{enumerate}
\item[$\square$] \emph{ ravine }
\begin{enumerate}
\item{
\begin{lstlisting}
Raven \Rav"en\ (r[a^]v"'n), n. [OF. ravine impetuosity,
   violence, F. ravine ravine. See {Ravine}, {Rapine}.] [Written
   also {ravin}, and {ravine}.]
   1. Rapine; rapacity. --Ray.
      [1913 Webster]
   2. Prey; plunder; food obtained by violence.
      [1913 Webster]
Raven \Rav"en\, v. i.
   To prey with rapacity; to be greedy; to show rapacity.
   [Written also {ravin}, and {ravine}.]
   [1913 Webster]
         Benjamin shall raven as a wolf.          --Gen. xlix.
                                                  27.
   [1913 Webster]
\end{lstlisting}}
\end{enumerate}
\item[$\square$] \emph{ rapacity }
\begin{enumerate}
\item{
\begin{lstlisting}
Rapacity \Ra*pac"i*ty\ (r[.a]*p[a^]s"[i^]*t[y^]), n. [L.
   rapacitas: cf. F. rapacit['e]. See {Rapacious}.]
   1. The quality of being rapacious; rapaciousness;
      ravenousness; as, the rapacity of pirates; the rapacity of
      wolves.
      [1913 Webster]
   2. The act or practice of extorting or exacting by oppressive
      injustice; exorbitant greediness of gain. ``The rapacity
      of some ages.'' --Sprat.
      [1913 Webster]
\end{lstlisting}}
\end{enumerate}
\item[$\square$] \emph{ contingent }
\begin{enumerate}
\item{
\begin{lstlisting}
Contingent \Con*tin"gent\, n.
   1. An event which may or may not happen; that which is
      unforeseen, undetermined, or dependent on something
      future; a contingency.
      [1913 Webster]
            His understanding could almost pierce into future
            contingents.                          --South.
      [1913 Webster]
   2. That which falls to one in a division or apportionment
      among a number; a suitable share; proportion; esp., a
      quota of troops.
      [1913 Webster]
            From the Alps to the border of Flanders, contingents
            were required . . . 200,000 men were in arms.
                                                  --Milman.
      [1913 Webster]
Contingent \Con*tin"gent\, a. [L. contingens, -entis, p. pr. of
   contingere to touch on all sides, to happen; con- + tangere
   to touch: cf. F. contingent. See {Tangent}, {Tact}.]
   1. Possible, or liable, but not certain, to occur;
      incidental; casual.
      [1913 Webster]
            Weighing so much actual crime against so much
            contingent advantage.                 --Burke.
      [1913 Webster]
   2. Dependent on that which is undetermined or unknown; as,
      the success of his undertaking is contingent upon events
      which he can not control. ``Uncertain and contingent
      causes.'' --Tillotson.
      [1913 Webster]
   3. (Law) Dependent for effect on something that may or may
      not occur; as, a contingent estate.
      [1913 Webster]
            If a contingent legacy be left to any one when he
            attains, or if he attains, the age of twenty-one.
                                                  --Blackstone.
      [1913 Webster]
\end{lstlisting}}
\end{enumerate}
\item[$\square$] \emph{ respite }
\begin{enumerate}
\item{
\begin{lstlisting}
Respite \Res"pite\ (r?s"p?t), n. [OF. respit, F. r['e]pit, from
   L. respectus respect, regard, delay, in LL., the deferring of
   a day. See {Respect}.]
   1. A putting off of that which was appointed; a postponement
      or delay.
      [1913 Webster]
            I crave but four day's respite.       --Shak.
      [1913 Webster]
   2. Temporary intermission of labor, or of any process or
      operation; interval of rest; pause; delay. ``Without more
      respite.'' --Chaucer.
      [1913 Webster]
            Some pause and respite only I require. --Denham.
      [1913 Webster]
   3. (Law)
      (a) Temporary suspension of the execution of a capital
          offender; reprieve.
      (b) The delay of appearance at court granted to a jury
          beyond the proper term.
          [1913 Webster]
   Syn: Pause; interval; stop; cessation; delay; postponement;
        stay; reprieve.
        [1913 Webster]
Respite \Res"pite\, v. t. [imp. & p. p. {Respited}; p. pr. & vb.
   n. {Respiting}.] [OF. respiter, LL. respectare. See
   {Respite}, n.]
   To give or grant a respite to. Specifically:
   (a) To delay or postpone; to put off.
   (b) To keep back from execution; to reprieve.
       [1913 Webster]
             Forty days longer we do respite you. --Shak.
       [1913 Webster]
   (c) To relieve by a pause or interval of rest. ``To respite
       his day labor with repast.'' --Milton.
       [1913 Webster]
\end{lstlisting}}
\end{enumerate}
\item[$\square$] \emph{ quixotic }
\begin{enumerate}
\item{
\begin{lstlisting}
Quixotic \Quix*ot"ic\ (kw[i^]ks*[o^]t"[i^]k), a.
   1. Like Don Quixote; romantic to extravagance; prone to
      pursue unrealizable goals; absurdly chivalric; apt to be
      deluded. See also {quixotism}. ``Feats of quixotic
      gallantry.'' --Prescott.
      [1913 Webster]
   2. Like the deeds of Don Quixote; ridiculously impractical;
      unachievable; extravagantly romantic; doomed to failure;
      as, a quixotic quest.
      [PJC]
            The word ``quixotic'' . . . has entered the common
            language, with the meaning ``hopelessly naive and
            idealistic,'' ``ridiculously impractical,'' ``doomed
            to fail.'' That this epithet can be used now in an
            exclusively pejorative sense not only shows that we
            have ceased to read Cervantes and to understand his
            character, but more fundamentally it reveals that
            our culture has drifted away from its spiritual
            roots.                                --Simon Leys
                                                  (N. Y. Review
                                                  of Books, June
                                                  11, 1998, p.
                                                  35)
      [PJC]
\end{lstlisting}}
\end{enumerate}
\item[$\square$] \emph{ dilettante }
\begin{enumerate}
\item{
\begin{lstlisting}
Dilettante \Dil`et*tan"te\, n.; pl. {Dilettanti}. [It., prop. p.
   pr. of dillettare to take delight in, fr. L. delectare to
   delight. See {Delight}, v. t.]
   An admirer or lover of the fine arts; popularly, an amateur;
   especially, one who follows an art or a branch of knowledge,
   desultorily, or for amusement only.
   [1913 Webster]
         The true poet is not an eccentric creature, not a mere
         artist living only for art, not a dreamer or a
         dilettante, sipping the nectar of existence, while he
         keeps aloof from its deeper interests.   --J. C.
                                                  Shairp.
   [1913 Webster]
\end{lstlisting}}
\end{enumerate}
\item[$\square$] \emph{ enamel }
\begin{enumerate}
\item{
\begin{lstlisting}
Enamel \En*am"el\, v. t. [imp. & p. p. {Enameled}or {Enamelled};
   p. pr. & vb. n. {Enameling} or {Enamelling}.]
   1. To lay enamel upon; to decorate with enamel whether inlaid
      or painted.
      [1913 Webster]
   2. To variegate with colors as if with enamel.
      [1913 Webster]
            Oft he [the serpent]bowed
            His turret crest and sleek enameled neck. --Milton.
      [1913 Webster]
   3. To form a glossy surface like enamel upon; as, to enamel
      card paper; to enamel leather or cloth.
      [1913 Webster]
   4. To disguise with cosmetics, as a woman's complexion.
      [1913 Webster]
Enamel \En*am"el\, v. i.
   To practice the art of enameling.
   [1913 Webster]
Enamel \En*am"el\, a.
   Relating to the art of enameling; as, enamel painting.
   --Tomlinson.
   [1913 Webster]
Enamel \En*am"el\, n. [Pref. en- + amel. See {Amel}, {Smelt}, v.
   t.]
   1. A variety of glass, used in ornament, to cover a surface,
      as of metal or pottery, and admitting of after decoration
      in color, or used itself for inlaying or application in
      varied colors.
      [1913 Webster]
   2. (Min.) A glassy, opaque bead obtained by the blowpipe.
      [1913 Webster]
   3. That which is enameled; also, any smooth, glossy surface,
      resembling enamel, especially if variegated.
      [1913 Webster]
   4. (Anat.) The intensely hard calcified tissue entering into
      the composition of teeth. It merely covers the exposed
      parts of the teeth of man, but in many animals is
      intermixed in various ways with the dentine and cement.
      [1913 Webster]
   5. Any one of various preparations for giving a smooth,
      glossy surface like that of enamel.
      [Webster 1913 Suppl.]
   6. A cosmetic intended to give the appearance of a smooth and
      beautiful complexion.
      [Webster 1913 Suppl.]
   {Enamel painting}, painting with enamel colors upon a ground
      of metal, porcelain, or the like, the colors being
      afterwards fixed by fire.
   {Enamel paper}, paper glazed a metallic coating.
      [1913 Webster]
\end{lstlisting}}
\end{enumerate}
\end{itemize}
\end{document}
