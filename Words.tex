\documentclass{article}

\usepackage{amssymb}
\usepackage[utf8]{inputenc}
\usepackage[english,russian]{babel}
\usepackage{alphabeta}
\usepackage{titling}
\usepackage[hidelinks]{hyperref}
\usepackage{listings}
\setlength{\droptitle}{-2cm}
\usepackage[a4paper,top=3cm,bottom=2cm,left=2.75cm,right=2.75cm,marginparwidth=1.75cm]{geometry}

\author{Θад Капэтанакис\\ {\small Computer Science Department, University of Athens}\\ {\tiny https://github.com/nlogocntrcaphnt}}
\title{My very first {\LaTeX} document: The Dictionary of Shame}
\date{\textit{Anno Domini, MMXXI}}

\lstset{
basicstyle=\small\rmfamily,
columns=fullflexible,
breaklines=true,
escapeinside={(*}{*)}
}

\begin{document}
	\maketitle
%	\renewcommand{\labelitemi}{$\blacksquare$}
%	\renewcommand\labelitemii{$\square$}
	\renewcommand{\labelenumi}{\Roman{enumi}}
	\begin{itemize}


\item[$\square$] \emph{ inoculate }
\begin{enumerate}
\item{
\begin{lstlisting}
Inoculate \In*oc"u*late\, v. t. [imp. & p. p. {Inoculated}; p.
   pr. & vb. n. {Inoculating}.] [L. inoculatus, p. p. of
   inoculare to ingraft; pref. in- in, on + oculare to furnish
   with eyes, fr. oculus an eye, also, a bud. See {Ocular}.]
   1. To bud; to insert, or graft, as the bud of a tree or plant
      in another tree or plant.
      [1913 Webster]
   2. To insert a foreign bud into; as, to inoculate a tree.
      [1913 Webster]
   3. (Med.) To communicate a disease to (a person) by inserting
      infectious matter in the skin or flesh, especially as a
      means of inducing immunological resistance to that or
      related diseases; as, to inoculate a person with the virus
      of smallpox, rabies, etc. See {Vaccinate}.
      [1913 Webster +PJC]
   4. Fig.: To introduce into the mind; -- used especially of
      harmful ideas or principles; to imbue; as, to inoculate
      one with treason or infidelity.
      [1913 Webster]
   5. (Microbiology) To introduce microorganisms into (a growth
      medium), to cause the growth and multiplication of the
      microorganisms; as, to inoculate a fermentation vat with
      an actinomycete culture in order to produce streptomycin.
      [PJC]
Inoculate \In*oc"u*late\, v. i.
   1. To graft by inserting buds.
      [1913 Webster]
   2. To communicate disease by inoculation.
      [1913 Webster]
\end{lstlisting}}
\end{enumerate}
\item[$\square$] \emph{ cadre }
\begin{enumerate}
\item{
\begin{lstlisting}
Cadre \Ca"dre\, n. [F. cadre, It. quadro square, from L.
   quadrum, fr. quatuor four.] (Mil.)
   The framework or skeleton upon which a regiment is to be
   formed; the officers of a regiment forming the staff.
   [Written also {cader}.]
   [1913 Webster]
\end{lstlisting}}
\end{enumerate}

\item[$\square$] \emph{ sardonic }
\begin{enumerate}
\item{
\begin{lstlisting}
Sardonic \Sar*don"ic\, a.
   Of, pertaining to, or resembling, a kind of linen made at
   Colchis.
   [1913 Webster]
Sardonic \Sar*don"ic\, a. [F. sardonique, L. sardonius, Gr. ?,
   ?, perhaps fr. ? to grin like a dog, or from a certain plant
   of Sardinia, Gr. ?, which was said to screw up the face of
   the eater.]
   Forced; unnatural; insincere; hence, derisive, mocking,
   malignant, or bitterly sarcastic; -- applied only to a laugh,
   smile, or some facial semblance of gayety.
   [1913 Webster]
         Where strained, sardonic smiles are glozing still,
         And grief is forced to laugh against her will. --Sir H.
                                                  Wotton.
   [1913 Webster]
         The scornful, ferocious, sardonic grin of a bloody
         ruffian.                                 --Burke.
   [1913 Webster]
   {Sardonic grin} or {Sardonic laugh}, an old medical term for
      a spasmodic affection of the muscles of the face, giving
      it an appearance of laughter.
      [1913 Webster]
\end{lstlisting}}
\end{enumerate}
\item[$\square$] \emph{ dilettante }
\begin{enumerate}
\item{
\begin{lstlisting}
Dilettante \Dil`et*tan"te\, n.; pl. {Dilettanti}. [It., prop. p.
   pr. of dillettare to take delight in, fr. L. delectare to
   delight. See {Delight}, v. t.]
   An admirer or lover of the fine arts; popularly, an amateur;
   especially, one who follows an art or a branch of knowledge,
   desultorily, or for amusement only.
   [1913 Webster]
         The true poet is not an eccentric creature, not a mere
         artist living only for art, not a dreamer or a
         dilettante, sipping the nectar of existence, while he
         keeps aloof from its deeper interests.   --J. C.
                                                  Shairp.
   [1913 Webster]
\end{lstlisting}}
\end{enumerate}
\item[$\square$] \emph{ petulant }
\begin{enumerate}
\item{
\begin{lstlisting}
Petulant \Pet"u*lant\, a. [L. petulans, -antis, prop., making
   slight attacks upon, from a lost dim. of petere to fall upon,
   to attack: cf. F. p['e]tulant. See {Petition}.]
   1. Forward; pert; insolent; wanton. [Obs.] --Burton.
      [1913 Webster]
   2. Capriciously fretful; characterized by ill-natured
      freakishness; irritable. ``Petulant moods.'' --Macaulay.
      [1913 Webster]
   Syn: Irritable; ill-humored; peevish; cross; fretful;
        querulous.
        [1913 Webster]
\end{lstlisting}}
\end{enumerate}
\item[$\square$] \emph{ infamia }
\begin{enumerate}
\item{
\begin{lstlisting}
Infamy \In"fa*my\, n.; pl. {Infamies}. [L. infamia, fr. infamis
   infamous; pref. in- not + fama fame: cf. F. infamie. See
   {Fame}.]
   [1913 Webster]
   1. Total loss of reputation; public disgrace; dishonor;
      ignominy; indignity.
      [1913 Webster]
            The afflicted queen would not yield, and said she
            would not . . . submit to such infamy. --Bp. Burnet.
      [1913 Webster]
   2. A quality which exposes to disgrace; extreme baseness or
      vileness; as, the infamy of an action.
      [1913 Webster]
   3. (Law) That loss of character, or public disgrace, which a
      convict incurs, and by which he is at common law rendered
      incompetent as a witness.
      [1913 Webster]
            Yesterday, Dec. 7, 1941 -- a day which will live in
            infamy, . . .                         --Franklin D.
                                                  Roosevelt.
\end{lstlisting}}
\end{enumerate}
\item[$\square$] \emph{ ninny }
\begin{enumerate}
\item{
\begin{lstlisting}
Ninny \Nin"ny\, n.; pl. {Ninnies}. [Cf. It. ninno, ninna, a
   baby, Sp. ni[~n]o, ni[~n]a, child, infant, It. ninna, ninna
   nanna, lullably, prob. fr. ni, na, as used in singing a child
   to sleep.]
   A fool; a simpleton. --Shak.
   [1913 Webster]
\end{lstlisting}}
\end{enumerate}
\item[$\square$] \emph{ soykaf }
\begin{enumerate}
\item{
\begin{lstlisting}
Soakage \Soak"age\, n.
   The act of soaking, or the state of being soaked; also, the
   quantity that enters or issues by soaking.
   [1913 Webster]
\end{lstlisting}}
\end{enumerate}
\item[$\square$] \emph{ invidia }
\begin{enumerate}
\item{
\begin{lstlisting}
Insidiate \In*sid"i*ate\, v. t. [L. insidiatus, p. p. of
   insidiare to lie in ambush, fr. insidiae. See {Insidious}.]
   To lie in ambush for. [Obs.] --Heywood.
   [1913 Webster]
\end{lstlisting}}
\end{enumerate}
\item[$\square$] \emph{ commiseration }
\begin{enumerate}
\item{
\begin{lstlisting}
Commiseration \Com*mis`er*a"tion\, n. [F. commis['e]ration, fr.
   L. commiseratio a part of an oration intended to excite
   compassion.]
   The act of commiserating; sorrow for the wants, afflictions,
   or distresses of another; pity; compassion.
   [1913 Webster]
         And pluck commiseration of his state
         From brassy bosoms and rough hearts of flint. --Shak.
   Syn: See {Sympathy}.
        [1913 Webster]
\end{lstlisting}}
\end{enumerate}
\end{itemize}
\end{document}
