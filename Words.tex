\documentclass{article}

\usepackage{amssymb}
\usepackage[utf8]{inputenc}
\usepackage[english,russian]{babel}
\usepackage{alphabeta}
\usepackage{titling}
\usepackage[hidelinks]{hyperref}
\usepackage{listings}
\setlength{\droptitle}{-2cm}
\usepackage[a4paper,top=3cm,bottom=2cm,left=2.75cm,right=2.75cm,marginparwidth=1.75cm]{geometry}

\author{Θад Капэтанакис\\ {\small Computer Science Department, University of Athens}\\ {\tiny https://github.com/nlogocntrcaphnt}}
\title{My very first {\LaTeX} document: The Dictionary of Shame}
\date{\textit{Anno Domini, MMXXI}}

\lstset{
basicstyle=\small\rmfamily,
columns=fullflexible,
breaklines=true,
escapeinside={(*}{*)}
}

\begin{document}
	\maketitle
%	\renewcommand{\labelitemi}{$\blacksquare$}
%	\renewcommand\labelitemii{$\square$}
	\renewcommand{\labelenumi}{\Roman{enumi}}
	\begin{itemize}


\item[$\square$] \emph{ inoculate }
\begin{enumerate}
\item{
\begin{lstlisting}
Inoculate \In*oc"u*late\, v. t. [imp. & p. p. {Inoculated}; p.
   pr. & vb. n. {Inoculating}.] [L. inoculatus, p. p. of
   inoculare to ingraft; pref. in- in, on + oculare to furnish
   with eyes, fr. oculus an eye, also, a bud. See {Ocular}.]
   1. To bud; to insert, or graft, as the bud of a tree or plant
      in another tree or plant.
      [1913 Webster]
   2. To insert a foreign bud into; as, to inoculate a tree.
      [1913 Webster]
   3. (Med.) To communicate a disease to (a person) by inserting
      infectious matter in the skin or flesh, especially as a
      means of inducing immunological resistance to that or
      related diseases; as, to inoculate a person with the virus
      of smallpox, rabies, etc. See {Vaccinate}.
      [1913 Webster +PJC]
   4. Fig.: To introduce into the mind; -- used especially of
      harmful ideas or principles; to imbue; as, to inoculate
      one with treason or infidelity.
      [1913 Webster]
   5. (Microbiology) To introduce microorganisms into (a growth
      medium), to cause the growth and multiplication of the
      microorganisms; as, to inoculate a fermentation vat with
      an actinomycete culture in order to produce streptomycin.
      [PJC]
Inoculate \In*oc"u*late\, v. i.
   1. To graft by inserting buds.
      [1913 Webster]
   2. To communicate disease by inoculation.
      [1913 Webster]
\end{lstlisting}}
\end{enumerate}
\item[$\square$] \emph{ cadre }
\begin{enumerate}
\item{
\begin{lstlisting}
Cadre \Ca"dre\, n. [F. cadre, It. quadro square, from L.
   quadrum, fr. quatuor four.] (Mil.)
   The framework or skeleton upon which a regiment is to be
   formed; the officers of a regiment forming the staff.
   [Written also {cader}.]
   [1913 Webster]
\end{lstlisting}}
\end{enumerate}

\item[$\square$] \emph{ sardonic }
\begin{enumerate}
\item{
\begin{lstlisting}
Sardonic \Sar*don"ic\, a.
   Of, pertaining to, or resembling, a kind of linen made at
   Colchis.
   [1913 Webster]
Sardonic \Sar*don"ic\, a. [F. sardonique, L. sardonius, Gr. ?,
   ?, perhaps fr. ? to grin like a dog, or from a certain plant
   of Sardinia, Gr. ?, which was said to screw up the face of
   the eater.]
   Forced; unnatural; insincere; hence, derisive, mocking,
   malignant, or bitterly sarcastic; -- applied only to a laugh,
   smile, or some facial semblance of gayety.
   [1913 Webster]
         Where strained, sardonic smiles are glozing still,
         And grief is forced to laugh against her will. --Sir H.
                                                  Wotton.
   [1913 Webster]
         The scornful, ferocious, sardonic grin of a bloody
         ruffian.                                 --Burke.
   [1913 Webster]
   {Sardonic grin} or {Sardonic laugh}, an old medical term for
      a spasmodic affection of the muscles of the face, giving
      it an appearance of laughter.
      [1913 Webster]
\end{lstlisting}}
\end{enumerate}
\item[$\square$] \emph{ dilettante }
\begin{enumerate}
\item{
\begin{lstlisting}
Dilettante \Dil`et*tan"te\, n.; pl. {Dilettanti}. [It., prop. p.
   pr. of dillettare to take delight in, fr. L. delectare to
   delight. See {Delight}, v. t.]
   An admirer or lover of the fine arts; popularly, an amateur;
   especially, one who follows an art or a branch of knowledge,
   desultorily, or for amusement only.
   [1913 Webster]
         The true poet is not an eccentric creature, not a mere
         artist living only for art, not a dreamer or a
         dilettante, sipping the nectar of existence, while he
         keeps aloof from its deeper interests.   --J. C.
                                                  Shairp.
   [1913 Webster]
\end{lstlisting}}
\end{enumerate}
\item[$\square$] \emph{ petulant }
\begin{enumerate}
\item{
\begin{lstlisting}
Petulant \Pet"u*lant\, a. [L. petulans, -antis, prop., making
   slight attacks upon, from a lost dim. of petere to fall upon,
   to attack: cf. F. p['e]tulant. See {Petition}.]
   1. Forward; pert; insolent; wanton. [Obs.] --Burton.
      [1913 Webster]
   2. Capriciously fretful; characterized by ill-natured
      freakishness; irritable. ``Petulant moods.'' --Macaulay.
      [1913 Webster]
   Syn: Irritable; ill-humored; peevish; cross; fretful;
        querulous.
        [1913 Webster]
\end{lstlisting}}
\end{enumerate}
\item[$\square$] \emph{ infamia }
\begin{enumerate}
\item{
\begin{lstlisting}
Infamy \In"fa*my\, n.; pl. {Infamies}. [L. infamia, fr. infamis
   infamous; pref. in- not + fama fame: cf. F. infamie. See
   {Fame}.]
   [1913 Webster]
   1. Total loss of reputation; public disgrace; dishonor;
      ignominy; indignity.
      [1913 Webster]
            The afflicted queen would not yield, and said she
            would not . . . submit to such infamy. --Bp. Burnet.
      [1913 Webster]
   2. A quality which exposes to disgrace; extreme baseness or
      vileness; as, the infamy of an action.
      [1913 Webster]
   3. (Law) That loss of character, or public disgrace, which a
      convict incurs, and by which he is at common law rendered
      incompetent as a witness.
      [1913 Webster]
            Yesterday, Dec. 7, 1941 -- a day which will live in
            infamy, . . .                         --Franklin D.
                                                  Roosevelt.
\end{lstlisting}}
\end{enumerate}
\item[$\square$] \emph{ ninny }
\begin{enumerate}
\item{
\begin{lstlisting}
Ninny \Nin"ny\, n.; pl. {Ninnies}. [Cf. It. ninno, ninna, a
   baby, Sp. ni[~n]o, ni[~n]a, child, infant, It. ninna, ninna
   nanna, lullably, prob. fr. ni, na, as used in singing a child
   to sleep.]
   A fool; a simpleton. --Shak.
   [1913 Webster]
\end{lstlisting}}
\end{enumerate}
\item[$\square$] \emph{ soykaf }
\begin{enumerate}
\item{
\begin{lstlisting}
Soakage \Soak"age\, n.
   The act of soaking, or the state of being soaked; also, the
   quantity that enters or issues by soaking.
   [1913 Webster]
\end{lstlisting}}
\end{enumerate}
\item[$\square$] \emph{ invidia }
\begin{enumerate}
\item{
\begin{lstlisting}
Insidiate \In*sid"i*ate\, v. t. [L. insidiatus, p. p. of
   insidiare to lie in ambush, fr. insidiae. See {Insidious}.]
   To lie in ambush for. [Obs.] --Heywood.
   [1913 Webster]
\end{lstlisting}}
\end{enumerate}
\item[$\square$] \emph{ commiseration }
\begin{enumerate}
\item{
\begin{lstlisting}
Commiseration \Com*mis`er*a"tion\, n. [F. commis['e]ration, fr.
   L. commiseratio a part of an oration intended to excite
   compassion.]
   The act of commiserating; sorrow for the wants, afflictions,
   or distresses of another; pity; compassion.
   [1913 Webster]
         And pluck commiseration of his state
         From brassy bosoms and rough hearts of flint. --Shak.
   Syn: See {Sympathy}.
        [1913 Webster]
\end{lstlisting}}
\end{enumerate}
\item[$\square$] \emph{ samsarisation }
\begin{enumerate}
\item{
\begin{lstlisting}

\end{lstlisting}}
\end{enumerate}
\item[$\square$] \emph{ quack }
\begin{enumerate}
\item{
\begin{lstlisting}
Quack \Quack\, a.
   Pertaining to or characterized by, boasting and pretension;
   used by quacks; pretending to cure diseases; as, a quack
   medicine; a quack doctor.
   [1913 Webster]
Quack \Quack\, v. i. [imp. & p. p. {Qvacked}; p. pr. & vb. n.
   {Quacking}.] [Of imitative origin; cf. D. kwaken, G. quacken,
   quaken, Icel. kvaka to twitter.]
   [1913 Webster]
   1. To utter a sound like the cry of a duck.
      [1913 Webster]
   2. To make vain and loud pretensions; to boast. `` To quack
      of universal cures.'' --Hudibras.
      [1913 Webster]
   3. To act the part of a quack, or pretender.
      [1913 Webster]
Quack \Quack\, n.
   1. The cry of the duck, or a sound in imitation of it; a
      hoarse, quacking noise. --Chaucer.
      [1913 Webster]
   2. [Cf. {Quacksalver}.] A boastful pretender to medical
      skill; an empiric; an ignorant practitioner.
      [1913 Webster]
   3. Hence, one who boastfully pretends to skill or knowledge
      of any kind not possessed; a charlatan.
      [1913 Webster]
            Quacks political; quacks scientific, academical.
                                                  --Carlyle.
      [1913 Webster]
\end{lstlisting}}
\end{enumerate}
\item[$\square$] \emph{ continence }
\begin{enumerate}
\item{
\begin{lstlisting}
Continence \Con"ti*nence\, Continency \Con"ti*nen*cy\, n. [F.
   continence, L. continentia. See {Continent}, and cf.
   {Countenance}.]
   1. Self-restraint; self-command.
      [1913 Webster]
            He knew what to say; he knew also, when to leave
            off, -- a continence which is practiced by few
            writers.                              --Dryden.
      [1913 Webster]
   2. The restraint which a person imposes upon his desires and
      passions; the act or power of refraining from indulgence
      of the sexual appetite, esp. from unlawful indulgence;
      sometimes, moderation in sexual indulgence.
      [1913 Webster]
            If they [the unmarried and widows] have not
            continency, let them marry.           --1 Cor. vii.
                                                  9 (Rev. Ver.
                                                  ).
      [1913 Webster]
            Chastity is either abstinence or continence:
            abstinence is that of virgins or widows; continence,
            that of married persons.              --Jer. Taylor.
      [1913 Webster]
   3. Uninterrupted course; continuity. [Obs.] --Ayliffe.
      [1913 Webster]
\end{lstlisting}}
\end{enumerate}
\item[$\square$] \emph{ encomium }
\begin{enumerate}
\item{
\begin{lstlisting}
Encomium \En*co"mi*um\, n.; pl. {Encomiums}. [NL., fr. Gr. ? (a
   song) chanted in a Bacchic festival in praise of the god; ?
   in + ? a jovial festivity, revel. See {Comedy}.]
   Warm or high praise; panegyric; strong commendation.
   [1913 Webster]
         His encomiums awakened all my ardor.     --W. Irving.
   Syn: See {Eulogy}.
        [1913 Webster]
\end{lstlisting}}
\end{enumerate}
\item[$\square$] \emph{ ancillary }
\begin{enumerate}
\item{
\begin{lstlisting}
Ancillary \An"cil*la*ry\, a. [L. ancillaris, fr. ancilla a
   female servant.]
   Subservient or subordinate, like a handmaid; auxiliary.
   [1913 Webster]
         The Convocation of York seems to have been always
         considered as inferior, and even ancillary, to the
         greater province.                        --Hallam.
   [1913 Webster]
\end{lstlisting}}
\end{enumerate}
\item[$\square$] \emph{ sleuth }
\begin{enumerate}
\item{
\begin{lstlisting}
Sleuth \Sleuth\, n. [Icel. sl[=o][eth]. See {Slot} a track.]
   The track of man or beast as followed by the scent. [Scot.]
   --Halliwell.
   [1913 Webster]
\end{lstlisting}}
\end{enumerate}
\item[$\square$] \emph{ seminal }
\begin{enumerate}
\item{
\begin{lstlisting}
Seminal \Sem"i*nal\, a. [L. seminalis, fr. semen, seminis, seed,
   akin to serere to sow: cf. F. seminal. See {Sow} to scatter
   seed.]
   1. Pertaining to, containing, or consisting of, seed or
      semen; as, the seminal fluid.
      [1913 Webster]
   2. Contained in seed.
      [1913 Webster]
   3. Hence: Holding the first place in a series of developed
      results or consequents; serving as a source, or first
      principle; giving rise to related ideas or results;
      germinal; radical; primary; original; as, seminal
      principles of generation; seminal virtue; a seminal
      discovery.
      [1913 Webster +PJC]
            The idea of God is, beyond all question or
            comparison, the one great seminal principle. --Hare.
      [1913 Webster]
   {Seminal leaf} (Bot.), a seed leaf, or cotyleden.
   {Seminal receptacle}. (Zo["o]l.) Same as {Spermatheca}.
      [1913 Webster]
Seminal \Sem"i*nal\, n.
   A seed. [Obs.] --Sir T. Browne.
   [1913 Webster]
\end{lstlisting}}
\end{enumerate}
\item[$\square$] \emph{ laudatory }
\begin{enumerate}
\item{
\begin{lstlisting}
Laudatory \Laud"a*to*ry\, a. [L. laudatorius: cf. OF.
   laudatoire.]
   Of or pertaining praise, or to the expression of praise; as,
   laudatory verses; the laudatory powers of Dryden. --Sir J.
   Stephen.
   [1913 Webster]
\end{lstlisting}}
\end{enumerate}
\item[$\square$] \emph{ intrepid }
\begin{enumerate}
\item{
\begin{lstlisting}
Intrepid \In*trep"id\, a. [L. intrepidus: cf. F. intr['e]pide.
   See {In-} not, and {Trepidation}.]
   Not trembling or shaking with fear; fearless; bold; brave;
   undaunted; courageous; as, an intrepid soldier; intrepid
   spirit.
   Syn: Fearless; dauntless; resolute; brave; courageous;
        daring; valiant; heroic; doughty.
        [1913 Webster]
\end{lstlisting}}
\end{enumerate}
\item[$\square$] \emph{ confabulate }
\begin{enumerate}
\item{
\begin{lstlisting}
Confabulate \Con*fab"u*late\, v. i. [imp. & p. p.
   {Confabulated}; p. pr. & vb. n. {Confabulating}.] [L.
   confabulatus, p. p. of confabulary, to converse together;
   con- + fabulary to speak, fr. fabula. See {Fable}.]
   To talk familiarly together; to chat; to prattle.
   [1913 Webster]
         I shall not ask Jean Jaques Rousseau
         If birds confabulate or no.              --Cowper.
   [1913 Webster]
\end{lstlisting}}
\end{enumerate}
\item[$\square$] \emph{ cognoscenti }
\begin{enumerate}
\item{
\begin{lstlisting}
Cognoscente \Cog`nos*cen"te\, n.; pl. {Cognoscenti}. [OIt.
   cognoscente, p. pr. of cognoscere, It. conoscere to know.]
   A connoisseur. --Mason.
   [1913 Webster]
\end{lstlisting}}
\end{enumerate}
\item[$\square$] \emph{ ductility }
\begin{enumerate}
\item{
\begin{lstlisting}
Ductility \Duc*til"i*ty\, n. [Cf. F. ductilit['e].]
   1. The property of a metal which allows it to be drawn into
      wires or filaments.
      [1913 Webster]
   2. Tractableness; pliableness. --South.
      [1913 Webster]
\end{lstlisting}}
\end{enumerate}
\item[$\square$] \emph{ gait }
\begin{enumerate}
\item{
\begin{lstlisting}
Gait \Gait\, n. [See {Gate} a way.]
   1. A going; a walk; a march; a way.
      [1913 Webster]
            Good gentleman, go your gait, and let poor folks
            pass.                                 --Shak.
      [1913 Webster]
   2. Manner of walking or stepping; bearing or carriage while
      moving.
      [1913 Webster]
            'T is Cinna; I do know him by his gait. --Shak.
      [1913 Webster]
\end{lstlisting}}
\end{enumerate}
\item[$\square$] \emph{ flocculent }
\begin{enumerate}
\item{
\begin{lstlisting}
Flocculent \Floc"cu*lent\, a. [See {Flock} of wool.]
   1. Clothed with small flocks or flakes; woolly. --Gray.
      [1913 Webster]
   2. (Zo["o]l.) Applied to the down of newly hatched or
      unfledged birds.
      [1913 Webster]
   3. (Chem.) Having a structure like shredded wool, as some
      precipitates.
      [Webster 1913 Suppl.]
\end{lstlisting}}
\end{enumerate}
\item[$\square$] \emph{ oeuvre }
\begin{enumerate}
\item{
\begin{lstlisting}
Louver \Lou"ver\, Louvre \Lou"vre\, n. [OE. lover, OF. lover,
   lovier; or l'ouvert the opening, fr. overt, ouvert, p. p. of
   ovrir, ouvrir, to open, F. ouvrir. Cf. {Overt}.] (Arch.)
   A small lantern. See {Lantern}, 2
   (a) . [Written also {lover}, {loover}, {lovery}, and
       {luffer}.]
   2. Same as {louver boards}, below
      [PJC]
   3. A set of slats resembling louver boards, arranged in a
      vertical row and attached at each slat end to a frame
      inserted in or part of a door or window; the slats may be
      made of wood, plastic, or metal, and the angle of
      inclination of the slats may be adjustable simultaneously,
      to allow more or less light or air into the enclosure.
      [PJC]
   {Louver boards} or {Louver boarding}, the sloping boards set
      to shed rainwater outward in openings which are to be left
      otherwise unfilled; as belfry windows, the openings of a
      louver, etc.
   {Louver work}, slatted work.
\end{lstlisting}}
\end{enumerate}
\item[$\square$] \emph{ enmity }
\begin{enumerate}
\item{
\begin{lstlisting}
Enmity \En"mi*ty\, n.; pl. {Enmities}. [OE. enemyte, fr. enemy:
   cf. F. inimiti['e], OF. enemisti['e]. See {Enemy}, and cf.
   {Amity}.]
   1. The quality of being an enemy; hostile or unfriendly
      disposition.
      [1913 Webster]
            No ground of enmity between us known. --Milton.
      [1913 Webster]
   2. A state of opposition; hostility.
      [1913 Webster]
            The friendship of the world is enmity with God.
                                                  --James iv. 4.
   Syn: Rancor; hostility; hatred; aversion; antipathy;
        repugnance; animosity; ill will; malice; malevolence.
        See {Animosity}, {Rancor}.
        [1913 Webster]
\end{lstlisting}}
\end{enumerate}
\item[$\square$] \emph{ trepidation }
\begin{enumerate}
\item{
\begin{lstlisting}
Trepidation \Trep`i*da"tion\, n. [F. tr['e]pidation, L.
   trepidatio, fr. trepidare to hurry with alarm, to tremble,
   from trepidus agitated, disturbed, alarmed; cf. trepit he
   turns, Gr. ? to turn, E. torture.]
   1. An involuntary trembling, sometimes an effect of
      paralysis, but usually caused by terror or fear; quaking;
      quivering.
      [1913 Webster]
   2. Hence, a state of terror or alarm; fear; confusion;
      fright; as, the men were in great trepidation.
      [1913 Webster]
   3. (Anc. Astron.) A libration of the starry sphere in the
      Ptolemaic system; a motion ascribed to the firmament, to
      account for certain small changes in the position of the
      ecliptic and of the stars.
      [1913 Webster]
   Syn: Tremor; agitation; disturbance; fear.
        [1913 Webster]
\end{lstlisting}}
\end{enumerate}
\item[$\square$] \emph{ trounce }
\begin{enumerate}
\item{
\begin{lstlisting}
Trounce \Trounce\, v. t. [imp. & p. p. {Trounced}; p. pr. & vb.
   n. {Trouncing}.] [F. tronce, tronche, a stump, piece of wood.
   See {Truncheon}.]
   To punish or beat severely; to whip smartly; to flog; to
   castigate. [Colloq.]
   [1913 Webster]
\end{lstlisting}}
\end{enumerate}
\end{itemize}
\end{document}
